\documentclass[11pt]{article}

    \usepackage[breakable]{tcolorbox}
    \usepackage{parskip} % Stop auto-indenting (to mimic markdown behaviour)
    

    % Basic figure setup, for now with no caption control since it's done
    % automatically by Pandoc (which extracts ![](path) syntax from Markdown).
    \usepackage{graphicx}
    % Maintain compatibility with old templates. Remove in nbconvert 6.0
    \let\Oldincludegraphics\includegraphics
    % Ensure that by default, figures have no caption (until we provide a
    % proper Figure object with a Caption API and a way to capture that
    % in the conversion process - todo).
    \usepackage{caption}
    \DeclareCaptionFormat{nocaption}{}
    \captionsetup{format=nocaption,aboveskip=0pt,belowskip=0pt}

    \usepackage{float}
    \floatplacement{figure}{H} % forces figures to be placed at the correct location
    \usepackage{xcolor} % Allow colors to be defined
    \usepackage{enumerate} % Needed for markdown enumerations to work
    \usepackage{geometry} % Used to adjust the document margins
    \usepackage{amsmath} % Equations
    \usepackage{amssymb} % Equations
    \usepackage{textcomp} % defines textquotesingle
    % Hack from http://tex.stackexchange.com/a/47451/13684:
    \AtBeginDocument{%
        \def\PYZsq{\textquotesingle}% Upright quotes in Pygmentized code
    }
    \usepackage{upquote} % Upright quotes for verbatim code
    \usepackage{eurosym} % defines \euro

    \usepackage{iftex}
    \ifPDFTeX
        \usepackage[T1]{fontenc}
        \IfFileExists{alphabeta.sty}{
              \usepackage{alphabeta}
          }{
              \usepackage[mathletters]{ucs}
              \usepackage[utf8x]{inputenc}
          }
    \else
        \usepackage{fontspec}
        \usepackage{unicode-math}
    \fi

    \usepackage{fancyvrb} % verbatim replacement that allows latex
    \usepackage{grffile} % extends the file name processing of package graphics
                         % to support a larger range
    \makeatletter % fix for old versions of grffile with XeLaTeX
    \@ifpackagelater{grffile}{2019/11/01}
    {
      % Do nothing on new versions
    }
    {
      \def\Gread@@xetex#1{%
        \IfFileExists{"\Gin@base".bb}%
        {\Gread@eps{\Gin@base.bb}}%
        {\Gread@@xetex@aux#1}%
      }
    }
    \makeatother
    \usepackage[Export]{adjustbox} % Used to constrain images to a maximum size
    \adjustboxset{max size={0.9\linewidth}{0.9\paperheight}}

    % The hyperref package gives us a pdf with properly built
    % internal navigation ('pdf bookmarks' for the table of contents,
    % internal cross-reference links, web links for URLs, etc.)
    \usepackage{hyperref}
    % The default LaTeX title has an obnoxious amount of whitespace. By default,
    % titling removes some of it. It also provides customization options.
    \usepackage{titling}
    \usepackage{longtable} % longtable support required by pandoc >1.10
    \usepackage{booktabs}  % table support for pandoc > 1.12.2
    \usepackage{array}     % table support for pandoc >= 2.11.3
    \usepackage{calc}      % table minipage width calculation for pandoc >= 2.11.1
    \usepackage[inline]{enumitem} % IRkernel/repr support (it uses the enumerate* environment)
    \usepackage[normalem]{ulem} % ulem is needed to support strikethroughs (\sout)
                                % normalem makes italics be italics, not underlines
    \usepackage{soul}      % strikethrough (\st) support for pandoc >= 3.0.0
    \usepackage{mathrsfs}
    

    
    % Colors for the hyperref package
    \definecolor{urlcolor}{rgb}{0,.145,.698}
    \definecolor{linkcolor}{rgb}{.71,0.21,0.01}
    \definecolor{citecolor}{rgb}{.12,.54,.11}

    % ANSI colors
    \definecolor{ansi-black}{HTML}{3E424D}
    \definecolor{ansi-black-intense}{HTML}{282C36}
    \definecolor{ansi-red}{HTML}{E75C58}
    \definecolor{ansi-red-intense}{HTML}{B22B31}
    \definecolor{ansi-green}{HTML}{00A250}
    \definecolor{ansi-green-intense}{HTML}{007427}
    \definecolor{ansi-yellow}{HTML}{DDB62B}
    \definecolor{ansi-yellow-intense}{HTML}{B27D12}
    \definecolor{ansi-blue}{HTML}{208FFB}
    \definecolor{ansi-blue-intense}{HTML}{0065CA}
    \definecolor{ansi-magenta}{HTML}{D160C4}
    \definecolor{ansi-magenta-intense}{HTML}{A03196}
    \definecolor{ansi-cyan}{HTML}{60C6C8}
    \definecolor{ansi-cyan-intense}{HTML}{258F8F}
    \definecolor{ansi-white}{HTML}{C5C1B4}
    \definecolor{ansi-white-intense}{HTML}{A1A6B2}
    \definecolor{ansi-default-inverse-fg}{HTML}{FFFFFF}
    \definecolor{ansi-default-inverse-bg}{HTML}{000000}

    % common color for the border for error outputs.
    \definecolor{outerrorbackground}{HTML}{FFDFDF}

    % commands and environments needed by pandoc snippets
    % extracted from the output of `pandoc -s`
    \providecommand{\tightlist}{%
      \setlength{\itemsep}{0pt}\setlength{\parskip}{0pt}}
    \DefineVerbatimEnvironment{Highlighting}{Verbatim}{commandchars=\\\{\}}
    % Add ',fontsize=\small' for more characters per line
    \newenvironment{Shaded}{}{}
    \newcommand{\KeywordTok}[1]{\textcolor[rgb]{0.00,0.44,0.13}{\textbf{{#1}}}}
    \newcommand{\DataTypeTok}[1]{\textcolor[rgb]{0.56,0.13,0.00}{{#1}}}
    \newcommand{\DecValTok}[1]{\textcolor[rgb]{0.25,0.63,0.44}{{#1}}}
    \newcommand{\BaseNTok}[1]{\textcolor[rgb]{0.25,0.63,0.44}{{#1}}}
    \newcommand{\FloatTok}[1]{\textcolor[rgb]{0.25,0.63,0.44}{{#1}}}
    \newcommand{\CharTok}[1]{\textcolor[rgb]{0.25,0.44,0.63}{{#1}}}
    \newcommand{\StringTok}[1]{\textcolor[rgb]{0.25,0.44,0.63}{{#1}}}
    \newcommand{\CommentTok}[1]{\textcolor[rgb]{0.38,0.63,0.69}{\textit{{#1}}}}
    \newcommand{\OtherTok}[1]{\textcolor[rgb]{0.00,0.44,0.13}{{#1}}}
    \newcommand{\AlertTok}[1]{\textcolor[rgb]{1.00,0.00,0.00}{\textbf{{#1}}}}
    \newcommand{\FunctionTok}[1]{\textcolor[rgb]{0.02,0.16,0.49}{{#1}}}
    \newcommand{\RegionMarkerTok}[1]{{#1}}
    \newcommand{\ErrorTok}[1]{\textcolor[rgb]{1.00,0.00,0.00}{\textbf{{#1}}}}
    \newcommand{\NormalTok}[1]{{#1}}

    % Additional commands for more recent versions of Pandoc
    \newcommand{\ConstantTok}[1]{\textcolor[rgb]{0.53,0.00,0.00}{{#1}}}
    \newcommand{\SpecialCharTok}[1]{\textcolor[rgb]{0.25,0.44,0.63}{{#1}}}
    \newcommand{\VerbatimStringTok}[1]{\textcolor[rgb]{0.25,0.44,0.63}{{#1}}}
    \newcommand{\SpecialStringTok}[1]{\textcolor[rgb]{0.73,0.40,0.53}{{#1}}}
    \newcommand{\ImportTok}[1]{{#1}}
    \newcommand{\DocumentationTok}[1]{\textcolor[rgb]{0.73,0.13,0.13}{\textit{{#1}}}}
    \newcommand{\AnnotationTok}[1]{\textcolor[rgb]{0.38,0.63,0.69}{\textbf{\textit{{#1}}}}}
    \newcommand{\CommentVarTok}[1]{\textcolor[rgb]{0.38,0.63,0.69}{\textbf{\textit{{#1}}}}}
    \newcommand{\VariableTok}[1]{\textcolor[rgb]{0.10,0.09,0.49}{{#1}}}
    \newcommand{\ControlFlowTok}[1]{\textcolor[rgb]{0.00,0.44,0.13}{\textbf{{#1}}}}
    \newcommand{\OperatorTok}[1]{\textcolor[rgb]{0.40,0.40,0.40}{{#1}}}
    \newcommand{\BuiltInTok}[1]{{#1}}
    \newcommand{\ExtensionTok}[1]{{#1}}
    \newcommand{\PreprocessorTok}[1]{\textcolor[rgb]{0.74,0.48,0.00}{{#1}}}
    \newcommand{\AttributeTok}[1]{\textcolor[rgb]{0.49,0.56,0.16}{{#1}}}
    \newcommand{\InformationTok}[1]{\textcolor[rgb]{0.38,0.63,0.69}{\textbf{\textit{{#1}}}}}
    \newcommand{\WarningTok}[1]{\textcolor[rgb]{0.38,0.63,0.69}{\textbf{\textit{{#1}}}}}


    % Define a nice break command that doesn't care if a line doesn't already
    % exist.
    \def\br{\hspace*{\fill} \\* }
    % Math Jax compatibility definitions
    \def\gt{>}
    \def\lt{<}
    \let\Oldtex\TeX
    \let\Oldlatex\LaTeX
    \renewcommand{\TeX}{\textrm{\Oldtex}}
    \renewcommand{\LaTeX}{\textrm{\Oldlatex}}
    % Document parameters
    % Document title
    \title{proj1a}
    
    
    
    
    
    
    
% Pygments definitions
\makeatletter
\def\PY@reset{\let\PY@it=\relax \let\PY@bf=\relax%
    \let\PY@ul=\relax \let\PY@tc=\relax%
    \let\PY@bc=\relax \let\PY@ff=\relax}
\def\PY@tok#1{\csname PY@tok@#1\endcsname}
\def\PY@toks#1+{\ifx\relax#1\empty\else%
    \PY@tok{#1}\expandafter\PY@toks\fi}
\def\PY@do#1{\PY@bc{\PY@tc{\PY@ul{%
    \PY@it{\PY@bf{\PY@ff{#1}}}}}}}
\def\PY#1#2{\PY@reset\PY@toks#1+\relax+\PY@do{#2}}

\@namedef{PY@tok@w}{\def\PY@tc##1{\textcolor[rgb]{0.73,0.73,0.73}{##1}}}
\@namedef{PY@tok@c}{\let\PY@it=\textit\def\PY@tc##1{\textcolor[rgb]{0.24,0.48,0.48}{##1}}}
\@namedef{PY@tok@cp}{\def\PY@tc##1{\textcolor[rgb]{0.61,0.40,0.00}{##1}}}
\@namedef{PY@tok@k}{\let\PY@bf=\textbf\def\PY@tc##1{\textcolor[rgb]{0.00,0.50,0.00}{##1}}}
\@namedef{PY@tok@kp}{\def\PY@tc##1{\textcolor[rgb]{0.00,0.50,0.00}{##1}}}
\@namedef{PY@tok@kt}{\def\PY@tc##1{\textcolor[rgb]{0.69,0.00,0.25}{##1}}}
\@namedef{PY@tok@o}{\def\PY@tc##1{\textcolor[rgb]{0.40,0.40,0.40}{##1}}}
\@namedef{PY@tok@ow}{\let\PY@bf=\textbf\def\PY@tc##1{\textcolor[rgb]{0.67,0.13,1.00}{##1}}}
\@namedef{PY@tok@nb}{\def\PY@tc##1{\textcolor[rgb]{0.00,0.50,0.00}{##1}}}
\@namedef{PY@tok@nf}{\def\PY@tc##1{\textcolor[rgb]{0.00,0.00,1.00}{##1}}}
\@namedef{PY@tok@nc}{\let\PY@bf=\textbf\def\PY@tc##1{\textcolor[rgb]{0.00,0.00,1.00}{##1}}}
\@namedef{PY@tok@nn}{\let\PY@bf=\textbf\def\PY@tc##1{\textcolor[rgb]{0.00,0.00,1.00}{##1}}}
\@namedef{PY@tok@ne}{\let\PY@bf=\textbf\def\PY@tc##1{\textcolor[rgb]{0.80,0.25,0.22}{##1}}}
\@namedef{PY@tok@nv}{\def\PY@tc##1{\textcolor[rgb]{0.10,0.09,0.49}{##1}}}
\@namedef{PY@tok@no}{\def\PY@tc##1{\textcolor[rgb]{0.53,0.00,0.00}{##1}}}
\@namedef{PY@tok@nl}{\def\PY@tc##1{\textcolor[rgb]{0.46,0.46,0.00}{##1}}}
\@namedef{PY@tok@ni}{\let\PY@bf=\textbf\def\PY@tc##1{\textcolor[rgb]{0.44,0.44,0.44}{##1}}}
\@namedef{PY@tok@na}{\def\PY@tc##1{\textcolor[rgb]{0.41,0.47,0.13}{##1}}}
\@namedef{PY@tok@nt}{\let\PY@bf=\textbf\def\PY@tc##1{\textcolor[rgb]{0.00,0.50,0.00}{##1}}}
\@namedef{PY@tok@nd}{\def\PY@tc##1{\textcolor[rgb]{0.67,0.13,1.00}{##1}}}
\@namedef{PY@tok@s}{\def\PY@tc##1{\textcolor[rgb]{0.73,0.13,0.13}{##1}}}
\@namedef{PY@tok@sd}{\let\PY@it=\textit\def\PY@tc##1{\textcolor[rgb]{0.73,0.13,0.13}{##1}}}
\@namedef{PY@tok@si}{\let\PY@bf=\textbf\def\PY@tc##1{\textcolor[rgb]{0.64,0.35,0.47}{##1}}}
\@namedef{PY@tok@se}{\let\PY@bf=\textbf\def\PY@tc##1{\textcolor[rgb]{0.67,0.36,0.12}{##1}}}
\@namedef{PY@tok@sr}{\def\PY@tc##1{\textcolor[rgb]{0.64,0.35,0.47}{##1}}}
\@namedef{PY@tok@ss}{\def\PY@tc##1{\textcolor[rgb]{0.10,0.09,0.49}{##1}}}
\@namedef{PY@tok@sx}{\def\PY@tc##1{\textcolor[rgb]{0.00,0.50,0.00}{##1}}}
\@namedef{PY@tok@m}{\def\PY@tc##1{\textcolor[rgb]{0.40,0.40,0.40}{##1}}}
\@namedef{PY@tok@gh}{\let\PY@bf=\textbf\def\PY@tc##1{\textcolor[rgb]{0.00,0.00,0.50}{##1}}}
\@namedef{PY@tok@gu}{\let\PY@bf=\textbf\def\PY@tc##1{\textcolor[rgb]{0.50,0.00,0.50}{##1}}}
\@namedef{PY@tok@gd}{\def\PY@tc##1{\textcolor[rgb]{0.63,0.00,0.00}{##1}}}
\@namedef{PY@tok@gi}{\def\PY@tc##1{\textcolor[rgb]{0.00,0.52,0.00}{##1}}}
\@namedef{PY@tok@gr}{\def\PY@tc##1{\textcolor[rgb]{0.89,0.00,0.00}{##1}}}
\@namedef{PY@tok@ge}{\let\PY@it=\textit}
\@namedef{PY@tok@gs}{\let\PY@bf=\textbf}
\@namedef{PY@tok@gp}{\let\PY@bf=\textbf\def\PY@tc##1{\textcolor[rgb]{0.00,0.00,0.50}{##1}}}
\@namedef{PY@tok@go}{\def\PY@tc##1{\textcolor[rgb]{0.44,0.44,0.44}{##1}}}
\@namedef{PY@tok@gt}{\def\PY@tc##1{\textcolor[rgb]{0.00,0.27,0.87}{##1}}}
\@namedef{PY@tok@err}{\def\PY@bc##1{{\setlength{\fboxsep}{\string -\fboxrule}\fcolorbox[rgb]{1.00,0.00,0.00}{1,1,1}{\strut ##1}}}}
\@namedef{PY@tok@kc}{\let\PY@bf=\textbf\def\PY@tc##1{\textcolor[rgb]{0.00,0.50,0.00}{##1}}}
\@namedef{PY@tok@kd}{\let\PY@bf=\textbf\def\PY@tc##1{\textcolor[rgb]{0.00,0.50,0.00}{##1}}}
\@namedef{PY@tok@kn}{\let\PY@bf=\textbf\def\PY@tc##1{\textcolor[rgb]{0.00,0.50,0.00}{##1}}}
\@namedef{PY@tok@kr}{\let\PY@bf=\textbf\def\PY@tc##1{\textcolor[rgb]{0.00,0.50,0.00}{##1}}}
\@namedef{PY@tok@bp}{\def\PY@tc##1{\textcolor[rgb]{0.00,0.50,0.00}{##1}}}
\@namedef{PY@tok@fm}{\def\PY@tc##1{\textcolor[rgb]{0.00,0.00,1.00}{##1}}}
\@namedef{PY@tok@vc}{\def\PY@tc##1{\textcolor[rgb]{0.10,0.09,0.49}{##1}}}
\@namedef{PY@tok@vg}{\def\PY@tc##1{\textcolor[rgb]{0.10,0.09,0.49}{##1}}}
\@namedef{PY@tok@vi}{\def\PY@tc##1{\textcolor[rgb]{0.10,0.09,0.49}{##1}}}
\@namedef{PY@tok@vm}{\def\PY@tc##1{\textcolor[rgb]{0.10,0.09,0.49}{##1}}}
\@namedef{PY@tok@sa}{\def\PY@tc##1{\textcolor[rgb]{0.73,0.13,0.13}{##1}}}
\@namedef{PY@tok@sb}{\def\PY@tc##1{\textcolor[rgb]{0.73,0.13,0.13}{##1}}}
\@namedef{PY@tok@sc}{\def\PY@tc##1{\textcolor[rgb]{0.73,0.13,0.13}{##1}}}
\@namedef{PY@tok@dl}{\def\PY@tc##1{\textcolor[rgb]{0.73,0.13,0.13}{##1}}}
\@namedef{PY@tok@s2}{\def\PY@tc##1{\textcolor[rgb]{0.73,0.13,0.13}{##1}}}
\@namedef{PY@tok@sh}{\def\PY@tc##1{\textcolor[rgb]{0.73,0.13,0.13}{##1}}}
\@namedef{PY@tok@s1}{\def\PY@tc##1{\textcolor[rgb]{0.73,0.13,0.13}{##1}}}
\@namedef{PY@tok@mb}{\def\PY@tc##1{\textcolor[rgb]{0.40,0.40,0.40}{##1}}}
\@namedef{PY@tok@mf}{\def\PY@tc##1{\textcolor[rgb]{0.40,0.40,0.40}{##1}}}
\@namedef{PY@tok@mh}{\def\PY@tc##1{\textcolor[rgb]{0.40,0.40,0.40}{##1}}}
\@namedef{PY@tok@mi}{\def\PY@tc##1{\textcolor[rgb]{0.40,0.40,0.40}{##1}}}
\@namedef{PY@tok@il}{\def\PY@tc##1{\textcolor[rgb]{0.40,0.40,0.40}{##1}}}
\@namedef{PY@tok@mo}{\def\PY@tc##1{\textcolor[rgb]{0.40,0.40,0.40}{##1}}}
\@namedef{PY@tok@ch}{\let\PY@it=\textit\def\PY@tc##1{\textcolor[rgb]{0.24,0.48,0.48}{##1}}}
\@namedef{PY@tok@cm}{\let\PY@it=\textit\def\PY@tc##1{\textcolor[rgb]{0.24,0.48,0.48}{##1}}}
\@namedef{PY@tok@cpf}{\let\PY@it=\textit\def\PY@tc##1{\textcolor[rgb]{0.24,0.48,0.48}{##1}}}
\@namedef{PY@tok@c1}{\let\PY@it=\textit\def\PY@tc##1{\textcolor[rgb]{0.24,0.48,0.48}{##1}}}
\@namedef{PY@tok@cs}{\let\PY@it=\textit\def\PY@tc##1{\textcolor[rgb]{0.24,0.48,0.48}{##1}}}

\def\PYZbs{\char`\\}
\def\PYZus{\char`\_}
\def\PYZob{\char`\{}
\def\PYZcb{\char`\}}
\def\PYZca{\char`\^}
\def\PYZam{\char`\&}
\def\PYZlt{\char`\<}
\def\PYZgt{\char`\>}
\def\PYZsh{\char`\#}
\def\PYZpc{\char`\%}
\def\PYZdl{\char`\$}
\def\PYZhy{\char`\-}
\def\PYZsq{\char`\'}
\def\PYZdq{\char`\"}
\def\PYZti{\char`\~}
% for compatibility with earlier versions
\def\PYZat{@}
\def\PYZlb{[}
\def\PYZrb{]}
\makeatother


    % For linebreaks inside Verbatim environment from package fancyvrb.
    \makeatletter
        \newbox\Wrappedcontinuationbox
        \newbox\Wrappedvisiblespacebox
        \newcommand*\Wrappedvisiblespace {\textcolor{red}{\textvisiblespace}}
        \newcommand*\Wrappedcontinuationsymbol {\textcolor{red}{\llap{\tiny$\m@th\hookrightarrow$}}}
        \newcommand*\Wrappedcontinuationindent {3ex }
        \newcommand*\Wrappedafterbreak {\kern\Wrappedcontinuationindent\copy\Wrappedcontinuationbox}
        % Take advantage of the already applied Pygments mark-up to insert
        % potential linebreaks for TeX processing.
        %        {, <, #, %, $, ' and ": go to next line.
        %        _, }, ^, &, >, - and ~: stay at end of broken line.
        % Use of \textquotesingle for straight quote.
        \newcommand*\Wrappedbreaksatspecials {%
            \def\PYGZus{\discretionary{\char`\_}{\Wrappedafterbreak}{\char`\_}}%
            \def\PYGZob{\discretionary{}{\Wrappedafterbreak\char`\{}{\char`\{}}%
            \def\PYGZcb{\discretionary{\char`\}}{\Wrappedafterbreak}{\char`\}}}%
            \def\PYGZca{\discretionary{\char`\^}{\Wrappedafterbreak}{\char`\^}}%
            \def\PYGZam{\discretionary{\char`\&}{\Wrappedafterbreak}{\char`\&}}%
            \def\PYGZlt{\discretionary{}{\Wrappedafterbreak\char`\<}{\char`\<}}%
            \def\PYGZgt{\discretionary{\char`\>}{\Wrappedafterbreak}{\char`\>}}%
            \def\PYGZsh{\discretionary{}{\Wrappedafterbreak\char`\#}{\char`\#}}%
            \def\PYGZpc{\discretionary{}{\Wrappedafterbreak\char`\%}{\char`\%}}%
            \def\PYGZdl{\discretionary{}{\Wrappedafterbreak\char`\$}{\char`\$}}%
            \def\PYGZhy{\discretionary{\char`\-}{\Wrappedafterbreak}{\char`\-}}%
            \def\PYGZsq{\discretionary{}{\Wrappedafterbreak\textquotesingle}{\textquotesingle}}%
            \def\PYGZdq{\discretionary{}{\Wrappedafterbreak\char`\"}{\char`\"}}%
            \def\PYGZti{\discretionary{\char`\~}{\Wrappedafterbreak}{\char`\~}}%
        }
        % Some characters . , ; ? ! / are not pygmentized.
        % This macro makes them "active" and they will insert potential linebreaks
        \newcommand*\Wrappedbreaksatpunct {%
            \lccode`\~`\.\lowercase{\def~}{\discretionary{\hbox{\char`\.}}{\Wrappedafterbreak}{\hbox{\char`\.}}}%
            \lccode`\~`\,\lowercase{\def~}{\discretionary{\hbox{\char`\,}}{\Wrappedafterbreak}{\hbox{\char`\,}}}%
            \lccode`\~`\;\lowercase{\def~}{\discretionary{\hbox{\char`\;}}{\Wrappedafterbreak}{\hbox{\char`\;}}}%
            \lccode`\~`\:\lowercase{\def~}{\discretionary{\hbox{\char`\:}}{\Wrappedafterbreak}{\hbox{\char`\:}}}%
            \lccode`\~`\?\lowercase{\def~}{\discretionary{\hbox{\char`\?}}{\Wrappedafterbreak}{\hbox{\char`\?}}}%
            \lccode`\~`\!\lowercase{\def~}{\discretionary{\hbox{\char`\!}}{\Wrappedafterbreak}{\hbox{\char`\!}}}%
            \lccode`\~`\/\lowercase{\def~}{\discretionary{\hbox{\char`\/}}{\Wrappedafterbreak}{\hbox{\char`\/}}}%
            \catcode`\.\active
            \catcode`\,\active
            \catcode`\;\active
            \catcode`\:\active
            \catcode`\?\active
            \catcode`\!\active
            \catcode`\/\active
            \lccode`\~`\~
        }
    \makeatother

    \let\OriginalVerbatim=\Verbatim
    \makeatletter
    \renewcommand{\Verbatim}[1][1]{%
        %\parskip\z@skip
        \sbox\Wrappedcontinuationbox {\Wrappedcontinuationsymbol}%
        \sbox\Wrappedvisiblespacebox {\FV@SetupFont\Wrappedvisiblespace}%
        \def\FancyVerbFormatLine ##1{\hsize\linewidth
            \vtop{\raggedright\hyphenpenalty\z@\exhyphenpenalty\z@
                \doublehyphendemerits\z@\finalhyphendemerits\z@
                \strut ##1\strut}%
        }%
        % If the linebreak is at a space, the latter will be displayed as visible
        % space at end of first line, and a continuation symbol starts next line.
        % Stretch/shrink are however usually zero for typewriter font.
        \def\FV@Space {%
            \nobreak\hskip\z@ plus\fontdimen3\font minus\fontdimen4\font
            \discretionary{\copy\Wrappedvisiblespacebox}{\Wrappedafterbreak}
            {\kern\fontdimen2\font}%
        }%

        % Allow breaks at special characters using \PYG... macros.
        \Wrappedbreaksatspecials
        % Breaks at punctuation characters . , ; ? ! and / need catcode=\active
        \OriginalVerbatim[#1,codes*=\Wrappedbreaksatpunct]%
    }
    \makeatother

    % Exact colors from NB
    \definecolor{incolor}{HTML}{303F9F}
    \definecolor{outcolor}{HTML}{D84315}
    \definecolor{cellborder}{HTML}{CFCFCF}
    \definecolor{cellbackground}{HTML}{F7F7F7}

    % prompt
    \makeatletter
    \newcommand{\boxspacing}{\kern\kvtcb@left@rule\kern\kvtcb@boxsep}
    \makeatother
    \newcommand{\prompt}[4]{
        {\ttfamily\llap{{\color{#2}[#3]:\hspace{3pt}#4}}\vspace{-\baselineskip}}
    }
    

    
    % Prevent overflowing lines due to hard-to-break entities
    \sloppy
    % Setup hyperref package
    \hypersetup{
      breaklinks=true,  % so long urls are correctly broken across lines
      colorlinks=true,
      urlcolor=urlcolor,
      linkcolor=linkcolor,
      citecolor=citecolor,
      }
    % Slightly bigger margins than the latex defaults
    
    \geometry{verbose,tmargin=1in,bmargin=1in,lmargin=1in,rmargin=1in}
    
    

\begin{document}
    
    \maketitle
    
    

    
    \begin{tcolorbox}[breakable, size=fbox, boxrule=1pt, pad at break*=1mm,colback=cellbackground, colframe=cellborder]
\prompt{In}{incolor}{1}{\boxspacing}
\begin{Verbatim}[commandchars=\\\{\}]
\PY{c+c1}{\PYZsh{} Initialize Otter}
\PY{k+kn}{import} \PY{n+nn}{otter}
\PY{n}{grader} \PY{o}{=} \PY{n}{otter}\PY{o}{.}\PY{n}{Notebook}\PY{p}{(}\PY{l+s+s2}{\PYZdq{}}\PY{l+s+s2}{proj1a.ipynb}\PY{l+s+s2}{\PYZdq{}}\PY{p}{)}
\end{Verbatim}
\end{tcolorbox}

    \hypertarget{project-1a-exploring-cook-county-housing}{%
\section{Project 1A: Exploring Cook County
Housing}\label{project-1a-exploring-cook-county-housing}}

\hypertarget{due-date-thursday-october-13th-1159-pm-pdt}{%
\subsection{Due Date: Thursday, October 13th, 11:59 PM
PDT}\label{due-date-thursday-october-13th-1159-pm-pdt}}

\hypertarget{collaboration-policy}{%
\subsubsection{Collaboration Policy}\label{collaboration-policy}}

Data science is a collaborative activity. While you may talk with others
about the homework, we ask that you \textbf{write your solutions
individually}. If you do discuss the assignments with others please
\textbf{include their names} in the collaborators cell below.

    \textbf{Collaborators:} \emph{list names here}

    

    \hypertarget{introduction}{%
\subsection{Introduction}\label{introduction}}

This project explores what can be learned from an extensive housing data
set that is embedded in a dense social context in Cook County, Illinois.

Here in part A, we will guide you through some basic exploratory data
analysis (EDA) to understand the structure of the data. Next, you will
be adding a few new features to the dataset, while cleaning the data as
well in the process.

In part B, you will specify and fit a linear model for the purpose of
prediction. Finally, we will analyze the error of the model and
brainstorm ways to improve the model's performance.

\hypertarget{score-breakdown}{%
\subsection{Score Breakdown}\label{score-breakdown}}

\begin{longtable}[]{@{}lll@{}}
\toprule()
Question & Part & Points \\
\midrule()
\endhead
1 & 1 & 1 \\
1 & 2 & 1 \\
1 & 3 & 1 \\
1 & 4 & 1 \\
2 & 1 & 1 \\
2 & 2 & 1 \\
3 & 1 & 3 \\
3 & 2 & 1 \\
3 & 3 & 1 \\
4 & - & 2 \\
5 & 1 & 1 \\
5 & 2 & 2 \\
5 & 3 & 2 \\
6 & 1 & 1 \\
6 & 2 & 2 \\
6 & 3 & 1 \\
6 & 4 & 2 \\
6 & 5 & 1 \\
7 & 1 & 1 \\
7 & 2 & 2 \\
Total & - & 28 \\
\bottomrule()
\end{longtable}

    \begin{tcolorbox}[breakable, size=fbox, boxrule=1pt, pad at break*=1mm,colback=cellbackground, colframe=cellborder]
\prompt{In}{incolor}{2}{\boxspacing}
\begin{Verbatim}[commandchars=\\\{\}]
\PY{k+kn}{import} \PY{n+nn}{numpy} \PY{k}{as} \PY{n+nn}{np}

\PY{k+kn}{import} \PY{n+nn}{pandas} \PY{k}{as} \PY{n+nn}{pd}
\PY{k+kn}{from} \PY{n+nn}{pandas}\PY{n+nn}{.}\PY{n+nn}{api}\PY{n+nn}{.}\PY{n+nn}{types} \PY{k+kn}{import} \PY{n}{CategoricalDtype}

\PY{o}{\PYZpc{}}\PY{k}{matplotlib} inline
\PY{k+kn}{import} \PY{n+nn}{matplotlib}\PY{n+nn}{.}\PY{n+nn}{pyplot} \PY{k}{as} \PY{n+nn}{plt}
\PY{k+kn}{import} \PY{n+nn}{seaborn} \PY{k}{as} \PY{n+nn}{sns}

\PY{k+kn}{import} \PY{n+nn}{warnings}
\PY{n}{warnings}\PY{o}{.}\PY{n}{filterwarnings}\PY{p}{(}\PY{l+s+s2}{\PYZdq{}}\PY{l+s+s2}{ignore}\PY{l+s+s2}{\PYZdq{}}\PY{p}{)}

\PY{k+kn}{import} \PY{n+nn}{zipfile}
\PY{k+kn}{import} \PY{n+nn}{os}

\PY{k+kn}{from} \PY{n+nn}{ds100\PYZus{}utils} \PY{k+kn}{import} \PY{n}{run\PYZus{}linear\PYZus{}regression\PYZus{}test}

\PY{c+c1}{\PYZsh{} Plot settings}
\PY{n}{plt}\PY{o}{.}\PY{n}{rcParams}\PY{p}{[}\PY{l+s+s1}{\PYZsq{}}\PY{l+s+s1}{figure.figsize}\PY{l+s+s1}{\PYZsq{}}\PY{p}{]} \PY{o}{=} \PY{p}{(}\PY{l+m+mi}{12}\PY{p}{,} \PY{l+m+mi}{9}\PY{p}{)}
\PY{n}{plt}\PY{o}{.}\PY{n}{rcParams}\PY{p}{[}\PY{l+s+s1}{\PYZsq{}}\PY{l+s+s1}{font.size}\PY{l+s+s1}{\PYZsq{}}\PY{p}{]} \PY{o}{=} \PY{l+m+mi}{12}
\end{Verbatim}
\end{tcolorbox}

    \hypertarget{the-data}{%
\section{The Data}\label{the-data}}

The data set consists of over 500 thousand records from Cook County,
Illinois, the county where Chicago is located. The data set we will be
working with has 61 features in total; the 62nd is sales price, which
you will predict with linear regression in the next part of this
project. An explanation of each variable can be found in the included
\texttt{codebook.txt} file. Some of the columns have been filtered out
to ensure this assignment doesn't become overly long when dealing with
data cleaning and formatting.

The data are split into training and test sets with 204,792 and 68,264
observations, respectively, but we will only be working on the training
set for this part of the project.

Let's first extract the data from the \texttt{cook\_county\_data.zip}.
Notice we didn't leave the \texttt{csv} files directly in the directory
because they take up too much space without some prior compression.

    \begin{tcolorbox}[breakable, size=fbox, boxrule=1pt, pad at break*=1mm,colback=cellbackground, colframe=cellborder]
\prompt{In}{incolor}{3}{\boxspacing}
\begin{Verbatim}[commandchars=\\\{\}]
\PY{k}{with} \PY{n}{zipfile}\PY{o}{.}\PY{n}{ZipFile}\PY{p}{(}\PY{l+s+s1}{\PYZsq{}}\PY{l+s+s1}{cook\PYZus{}county\PYZus{}data.zip}\PY{l+s+s1}{\PYZsq{}}\PY{p}{)} \PY{k}{as} \PY{n}{item}\PY{p}{:}
    \PY{n}{item}\PY{o}{.}\PY{n}{extractall}\PY{p}{(}\PY{p}{)}
\end{Verbatim}
\end{tcolorbox}

    Let's load the training data.

    \begin{tcolorbox}[breakable, size=fbox, boxrule=1pt, pad at break*=1mm,colback=cellbackground, colframe=cellborder]
\prompt{In}{incolor}{4}{\boxspacing}
\begin{Verbatim}[commandchars=\\\{\}]
\PY{n}{training\PYZus{}data} \PY{o}{=} \PY{n}{pd}\PY{o}{.}\PY{n}{read\PYZus{}csv}\PY{p}{(}\PY{l+s+s2}{\PYZdq{}}\PY{l+s+s2}{cook\PYZus{}county\PYZus{}train.csv}\PY{l+s+s2}{\PYZdq{}}\PY{p}{,} \PY{n}{index\PYZus{}col}\PY{o}{=}\PY{l+s+s1}{\PYZsq{}}\PY{l+s+s1}{Unnamed: 0}\PY{l+s+s1}{\PYZsq{}}\PY{p}{)}
\end{Verbatim}
\end{tcolorbox}

    As a good sanity check, we should at least verify that the data shape
matches the description.

    \begin{tcolorbox}[breakable, size=fbox, boxrule=1pt, pad at break*=1mm,colback=cellbackground, colframe=cellborder]
\prompt{In}{incolor}{5}{\boxspacing}
\begin{Verbatim}[commandchars=\\\{\}]
\PY{c+c1}{\PYZsh{} 204792 observations and 62 features in training data}
\PY{k}{assert} \PY{n}{training\PYZus{}data}\PY{o}{.}\PY{n}{shape} \PY{o}{==} \PY{p}{(}\PY{l+m+mi}{204792}\PY{p}{,} \PY{l+m+mi}{62}\PY{p}{)}
\PY{c+c1}{\PYZsh{} Sale Price is provided in the training data}
\PY{k}{assert} \PY{l+s+s1}{\PYZsq{}}\PY{l+s+s1}{Sale Price}\PY{l+s+s1}{\PYZsq{}} \PY{o+ow}{in} \PY{n}{training\PYZus{}data}\PY{o}{.}\PY{n}{columns}\PY{o}{.}\PY{n}{values}
\end{Verbatim}
\end{tcolorbox}

    The next order of business is getting a feel for the variables in our
data. A more detailed description of each variable is included in
\texttt{codebook.txt} (in the same directory as this notebook).
\textbf{You should take some time to familiarize yourself with the
codebook before moving forward.}

Let's take a quick look at all the current columns in our training data.

    \begin{tcolorbox}[breakable, size=fbox, boxrule=1pt, pad at break*=1mm,colback=cellbackground, colframe=cellborder]
\prompt{In}{incolor}{6}{\boxspacing}
\begin{Verbatim}[commandchars=\\\{\}]
\PY{n}{training\PYZus{}data}\PY{o}{.}\PY{n}{columns}\PY{o}{.}\PY{n}{values}
\end{Verbatim}
\end{tcolorbox}

            \begin{tcolorbox}[breakable, size=fbox, boxrule=.5pt, pad at break*=1mm, opacityfill=0]
\prompt{Out}{outcolor}{6}{\boxspacing}
\begin{Verbatim}[commandchars=\\\{\}]
array(['PIN', 'Property Class', 'Neighborhood Code', 'Land Square Feet',
       'Town Code', 'Apartments', 'Wall Material', 'Roof Material',
       'Basement', 'Basement Finish', 'Central Heating', 'Other Heating',
       'Central Air', 'Fireplaces', 'Attic Type', 'Attic Finish',
       'Design Plan', 'Cathedral Ceiling', 'Construction Quality',
       'Site Desirability', 'Garage 1 Size', 'Garage 1 Material',
       'Garage 1 Attachment', 'Garage 1 Area', 'Garage 2 Size',
       'Garage 2 Material', 'Garage 2 Attachment', 'Garage 2 Area',
       'Porch', 'Other Improvements', 'Building Square Feet',
       'Repair Condition', 'Multi Code', 'Number of Commercial Units',
       'Estimate (Land)', 'Estimate (Building)', 'Deed No.', 'Sale Price',
       'Longitude', 'Latitude', 'Census Tract',
       'Multi Property Indicator', 'Modeling Group', 'Age', 'Use',
       "O'Hare Noise", 'Floodplain', 'Road Proximity', 'Sale Year',
       'Sale Quarter', 'Sale Half-Year', 'Sale Quarter of Year',
       'Sale Month of Year', 'Sale Half of Year', 'Most Recent Sale',
       'Age Decade', 'Pure Market Filter', 'Garage Indicator',
       'Neigborhood Code (mapping)', 'Town and Neighborhood',
       'Description', 'Lot Size'], dtype=object)
\end{Verbatim}
\end{tcolorbox}
        
    \begin{tcolorbox}[breakable, size=fbox, boxrule=1pt, pad at break*=1mm,colback=cellbackground, colframe=cellborder]
\prompt{In}{incolor}{7}{\boxspacing}
\begin{Verbatim}[commandchars=\\\{\}]
\PY{n}{training\PYZus{}data}\PY{p}{[}\PY{l+s+s1}{\PYZsq{}}\PY{l+s+s1}{Description}\PY{l+s+s1}{\PYZsq{}}\PY{p}{]}\PY{p}{[}\PY{l+m+mi}{0}\PY{p}{]}
\end{Verbatim}
\end{tcolorbox}

            \begin{tcolorbox}[breakable, size=fbox, boxrule=.5pt, pad at break*=1mm, opacityfill=0]
\prompt{Out}{outcolor}{7}{\boxspacing}
\begin{Verbatim}[commandchars=\\\{\}]
'This property, sold on 09/14/2015, is a one-story houeshold located at 2950 S
LYMAN ST.It has a total of 6 rooms, 3 of which are bedrooms, and 1.0 of which
are bathrooms.'
\end{Verbatim}
\end{tcolorbox}
        
    \hypertarget{part-1-contextualizing-the-data}{%
\section{Part 1: Contextualizing the
Data}\label{part-1-contextualizing-the-data}}

Let's try to understand the background of our dataset before diving into
a full-scale analysis.

\hypertarget{question-1}{%
\subsection{Question 1}\label{question-1}}

\hypertarget{part-1}{%
\subsubsection{Part 1}\label{part-1}}

Based on the columns present in this data set and the values that they
take, what do you think each row represents? That is, what is the
granularity of this data set?

    Each row of the data set represents the data for one house. The
granularity of this data set is extremely fine as it showcases the all
the essential information such as the area of the each room in the
property, details about its location and proximity to major roads.

    \begin{center}\rule{0.5\linewidth}{0.5pt}\end{center}

\hypertarget{part-2}{%
\subsubsection{Part 2}\label{part-2}}

Why do you think this data was collected? For what purposes? By whom?

This question calls for your speculation and is looking for
thoughtfulness, not correctness.

    I believe this data was collected to evaluate the price of every
property taking all of its features in consideration to evaluate a
well-rounded price.

    \begin{center}\rule{0.5\linewidth}{0.5pt}\end{center}

\hypertarget{part-3}{%
\subsubsection{Part 3}\label{part-3}}

Certain variables in this data set contain information that either
directly contains demographic information (data on people) or could when
linked to other data sets. Identify at least one demographic-related
variable and explain the nature of the demographic data it embeds.

    The `in\_expensive\_neighborhood' column contains demographic
information as it tells us their financial stabilty becuase it tells
whether the person or family living there can afford the house in an
expensive neighborhood and if they can they are inclined to be wealthy
or financialy stable

    \begin{center}\rule{0.5\linewidth}{0.5pt}\end{center}

\hypertarget{part-4}{%
\subsubsection{Part 4}\label{part-4}}

Craft at least two questions about housing in Cook County that can be
answered with this data set and provide the type of analytical tool you
would use to answer it (e.g.~``I would create a \_\_\_ plot of \_\_\_
and \textbf{\emph{'' or ''I would calculate the }} {[}summary
statistic{]} for \_\_\_ and \_\_\_\_''). Be sure to reference the
columns that you would use and any additional data sets you would need
to answer that question.

    Q1) Is there any relationship between the construction quality of the
house and the age of the house? A) Since the `Construction Quality'
column is qualitative, I would create four histograms for all four
numbers denoting the construction quality where the frequency of the
property's age would on the y-axis and the age on the x-axis. I can
either overlay them on top of each other but that might get consfusing
so we can plot them seperately in the same figure, making them easier to
compare.

Q2) Is the sale price affected by the proximity to a major road? A) I
would plot 2 boxplots on the same figure one plot for properties close
to a major road and one plot for properties away from the mahor road
using the `Sale Price' and `Road Proximity' columns.

    \hypertarget{part-2-exploratory-data-analysis}{%
\section{Part 2: Exploratory Data
Analysis}\label{part-2-exploratory-data-analysis}}

This data set was collected by the
\href{https://datacatalog.cookcountyil.gov/Property-Taxation/Archive-Cook-County-Assessor-s-Residential-Sales-D/5pge-nu6u}{Cook
County Assessor's Office} in order to build a model to predict the
monetary value of a home (if you didn't put this for your answer for
Question 1 Part 2, please don't go back and change it - we wanted
speculation!). You can read more about data collection in the CCAO's
\href{https://gitlab.com/ccao-data-science---modeling/ccao_sf_cama_dev/-/blob/master/documentation/Preliminary\%20Report\%20on\%20Data\%20Integrity\%20June\%207,\%202019.pdf}{Residential
Data Integrity Preliminary Report}. In part 2 of this project you will
be building a linear model that predict sales prices using training data
but it's important to first understand how the structure of the data
informs such a model. In this section, we will make a series of
exploratory visualizations and feature engineering in preparation for
that prediction task.

Note that we will perform EDA on the \textbf{training data}.

\hypertarget{sale-price}{%
\subsubsection{Sale Price}\label{sale-price}}

We begin by examining the distribution of our target variable
\texttt{SalePrice}. At the same time, we also take a look at some
descriptive statistics of this variable. We have provided the following
helper method \texttt{plot\_distribution} that you can use to visualize
the distribution of the \texttt{SalePrice} using both the histogram and
the box plot at the same time. Run the following 2 cells and describe
what you think is wrong with the visualization.

    \begin{tcolorbox}[breakable, size=fbox, boxrule=1pt, pad at break*=1mm,colback=cellbackground, colframe=cellborder]
\prompt{In}{incolor}{8}{\boxspacing}
\begin{Verbatim}[commandchars=\\\{\}]
\PY{k}{def} \PY{n+nf}{plot\PYZus{}distribution}\PY{p}{(}\PY{n}{data}\PY{p}{,} \PY{n}{label}\PY{p}{)}\PY{p}{:}
    \PY{n}{fig}\PY{p}{,} \PY{n}{axs} \PY{o}{=} \PY{n}{plt}\PY{o}{.}\PY{n}{subplots}\PY{p}{(}\PY{n}{nrows}\PY{o}{=}\PY{l+m+mi}{2}\PY{p}{)}

    \PY{n}{sns}\PY{o}{.}\PY{n}{distplot}\PY{p}{(}
        \PY{n}{data}\PY{p}{[}\PY{n}{label}\PY{p}{]}\PY{p}{,} 
        \PY{n}{ax}\PY{o}{=}\PY{n}{axs}\PY{p}{[}\PY{l+m+mi}{0}\PY{p}{]}
    \PY{p}{)}
    \PY{n}{sns}\PY{o}{.}\PY{n}{boxplot}\PY{p}{(}
        \PY{n}{data}\PY{p}{[}\PY{n}{label}\PY{p}{]}\PY{p}{,}
        \PY{n}{width}\PY{o}{=}\PY{l+m+mf}{0.3}\PY{p}{,} 
        \PY{n}{ax}\PY{o}{=}\PY{n}{axs}\PY{p}{[}\PY{l+m+mi}{1}\PY{p}{]}\PY{p}{,}
        \PY{n}{showfliers}\PY{o}{=}\PY{k+kc}{False}\PY{p}{,}
    \PY{p}{)}

    \PY{c+c1}{\PYZsh{} Align axes}
    \PY{n}{spacer} \PY{o}{=} \PY{n}{np}\PY{o}{.}\PY{n}{max}\PY{p}{(}\PY{n}{data}\PY{p}{[}\PY{n}{label}\PY{p}{]}\PY{p}{)} \PY{o}{*} \PY{l+m+mf}{0.05}
    \PY{n}{xmin} \PY{o}{=} \PY{n}{np}\PY{o}{.}\PY{n}{min}\PY{p}{(}\PY{n}{data}\PY{p}{[}\PY{n}{label}\PY{p}{]}\PY{p}{)} \PY{o}{\PYZhy{}} \PY{n}{spacer}
    \PY{n}{xmax} \PY{o}{=} \PY{n}{np}\PY{o}{.}\PY{n}{max}\PY{p}{(}\PY{n}{data}\PY{p}{[}\PY{n}{label}\PY{p}{]}\PY{p}{)} \PY{o}{+} \PY{n}{spacer}
    \PY{n}{axs}\PY{p}{[}\PY{l+m+mi}{0}\PY{p}{]}\PY{o}{.}\PY{n}{set\PYZus{}xlim}\PY{p}{(}\PY{p}{(}\PY{n}{xmin}\PY{p}{,} \PY{n}{xmax}\PY{p}{)}\PY{p}{)}
    \PY{n}{axs}\PY{p}{[}\PY{l+m+mi}{1}\PY{p}{]}\PY{o}{.}\PY{n}{set\PYZus{}xlim}\PY{p}{(}\PY{p}{(}\PY{n}{xmin}\PY{p}{,} \PY{n}{xmax}\PY{p}{)}\PY{p}{)}

    \PY{c+c1}{\PYZsh{} Remove some axis text}
    \PY{n}{axs}\PY{p}{[}\PY{l+m+mi}{0}\PY{p}{]}\PY{o}{.}\PY{n}{xaxis}\PY{o}{.}\PY{n}{set\PYZus{}visible}\PY{p}{(}\PY{k+kc}{False}\PY{p}{)}
    \PY{n}{axs}\PY{p}{[}\PY{l+m+mi}{0}\PY{p}{]}\PY{o}{.}\PY{n}{yaxis}\PY{o}{.}\PY{n}{set\PYZus{}visible}\PY{p}{(}\PY{k+kc}{False}\PY{p}{)}
    \PY{n}{axs}\PY{p}{[}\PY{l+m+mi}{1}\PY{p}{]}\PY{o}{.}\PY{n}{yaxis}\PY{o}{.}\PY{n}{set\PYZus{}visible}\PY{p}{(}\PY{k+kc}{False}\PY{p}{)}

    \PY{c+c1}{\PYZsh{} Put the two plots together}
    \PY{n}{plt}\PY{o}{.}\PY{n}{subplots\PYZus{}adjust}\PY{p}{(}\PY{n}{hspace}\PY{o}{=}\PY{l+m+mi}{0}\PY{p}{)}
\end{Verbatim}
\end{tcolorbox}

    \begin{tcolorbox}[breakable, size=fbox, boxrule=1pt, pad at break*=1mm,colback=cellbackground, colframe=cellborder]
\prompt{In}{incolor}{9}{\boxspacing}
\begin{Verbatim}[commandchars=\\\{\}]
\PY{n}{plot\PYZus{}distribution}\PY{p}{(}\PY{n}{training\PYZus{}data}\PY{p}{,} \PY{n}{label}\PY{o}{=}\PY{l+s+s1}{\PYZsq{}}\PY{l+s+s1}{Sale Price}\PY{l+s+s1}{\PYZsq{}}\PY{p}{)}
\end{Verbatim}
\end{tcolorbox}

    \begin{center}
    \adjustimage{max size={0.9\linewidth}{0.9\paperheight}}{output_25_0.png}
    \end{center}
    { \hspace*{\fill} \\}
    
    \hypertarget{question-2}{%
\subsection{Question 2}\label{question-2}}

\hypertarget{part-1}{%
\subsubsection{Part 1}\label{part-1}}

Identify one issue with the visualization above and briefly describe one
way to overcome it. You may also want to try running
\texttt{training\_data{[}\textquotesingle{}Sale\ Price\textquotesingle{}{]}.describe()}
in a different cell to see some specific summary statistics on the
distribution of the target variable. Make sure to delete the cell
afterwards as the autograder may not work otherwise.

    The x-axis on the visualization above has caused the boxplot and
histogram to be cramped up to the left depicting a false image of the
data as the plot is not scaled. Moreover, there is no y-axis for the
histogram which makes it difficult to depict whether the histogram is
using count or density. The a-xis is extended to capture the max value
which occurs at \(7.1 * 10^7\) but that value is an outlier and distorts
the accuracy of the rest of the plot.

    

    \begin{tcolorbox}[breakable, size=fbox, boxrule=1pt, pad at break*=1mm,colback=cellbackground, colframe=cellborder]
\prompt{In}{incolor}{10}{\boxspacing}
\begin{Verbatim}[commandchars=\\\{\}]
\PY{n}{training\PYZus{}data}\PY{p}{[}\PY{l+s+s1}{\PYZsq{}}\PY{l+s+s1}{Sale Price}\PY{l+s+s1}{\PYZsq{}}\PY{p}{]}\PY{o}{.}\PY{n}{describe}\PY{p}{(}\PY{p}{)}
\end{Verbatim}
\end{tcolorbox}

            \begin{tcolorbox}[breakable, size=fbox, boxrule=.5pt, pad at break*=1mm, opacityfill=0]
\prompt{Out}{outcolor}{10}{\boxspacing}
\begin{Verbatim}[commandchars=\\\{\}]
count    2.047920e+05
mean     2.451646e+05
std      3.628694e+05
min      1.000000e+00
25\%      4.520000e+04
50\%      1.750000e+05
75\%      3.120000e+05
max      7.100000e+07
Name: Sale Price, dtype: float64
\end{Verbatim}
\end{tcolorbox}
        
    \begin{center}\rule{0.5\linewidth}{0.5pt}\end{center}

\hypertarget{part-2}{%
\subsubsection{Part 2}\label{part-2}}

To zoom in on the visualization of most households, we will focus only
on a subset of \texttt{Sale\ Price} for this assignment. In addition, it
may be a good idea to apply log transformation to \texttt{Sale\ Price}.
In the cell below, reassign \texttt{training\_data} to a new dataframe
that is the same as the original one \textbf{except with the following
changes}:

\begin{itemize}
\tightlist
\item
  \texttt{training\_data} should contain only households whose price is
  at least \$500.
\item
  \texttt{training\_data} should contain a new \texttt{Log\ Sale\ Price}
  column that contains the log-transformed sale prices.
\end{itemize}

\textbf{Note}: This also implies from now on, our target variable in the
model will be the log transformed sale prices from the column
\texttt{Log\ Sale\ Price}.

\textbf{Note}: You should \textbf{NOT} remove the original column
\texttt{Sale\ Price} as it will be helpful for later questions.

\emph{To ensure that any error from this part does not propagate to
later questions, there will be no hidden test here.}

    \begin{tcolorbox}[breakable, size=fbox, boxrule=1pt, pad at break*=1mm,colback=cellbackground, colframe=cellborder]
\prompt{In}{incolor}{11}{\boxspacing}
\begin{Verbatim}[commandchars=\\\{\}]
\PY{n}{training\PYZus{}data}
\end{Verbatim}
\end{tcolorbox}

            \begin{tcolorbox}[breakable, size=fbox, boxrule=.5pt, pad at break*=1mm, opacityfill=0]
\prompt{Out}{outcolor}{11}{\boxspacing}
\begin{Verbatim}[commandchars=\\\{\}]
                   PIN  Property Class  Neighborhood Code  Land Square Feet  \textbackslash{}
0       17294100610000             203                 50            2500.0
1       13272240180000             202                120            3780.0
2       25221150230000             202                210            4375.0
3       10251130030000             203                220            4375.0
4       31361040550000             202                120            8400.0
{\ldots}                {\ldots}             {\ldots}                {\ldots}               {\ldots}
204787  25163010260000             202                321            4375.0
204788   5063010090000             204                 21           16509.0
204789  16333020150000             202                 90            3810.0
204790   9242030500000             203                 80            6650.0
204791  19102030080000             203                 30            2500.0

        Town Code  Apartments  Wall Material  Roof Material  Basement  \textbackslash{}
0              76         0.0            2.0            1.0       1.0
1              71         0.0            2.0            1.0       1.0
2              70         0.0            2.0            1.0       2.0
3              17         0.0            3.0            1.0       1.0
4              32         0.0            3.0            1.0       2.0
{\ldots}           {\ldots}         {\ldots}            {\ldots}            {\ldots}       {\ldots}
204787         72         0.0            2.0            1.0       1.0
204788         23         0.0            1.0            1.0       1.0
204789         15         0.0            2.0            1.0       1.0
204790         22         0.0            2.0            1.0       1.0
204791         72         0.0            1.0            1.0       1.0

        Basement Finish  {\ldots}  Sale Month of Year  Sale Half of Year  \textbackslash{}
0                   3.0  {\ldots}                   9                  2
1                   1.0  {\ldots}                   5                  1
2                   3.0  {\ldots}                   2                  1
3                   3.0  {\ldots}                   7                  2
4                   3.0  {\ldots}                   6                  1
{\ldots}                 {\ldots}  {\ldots}                 {\ldots}                {\ldots}
204787              1.0  {\ldots}                   7                  2
204788              1.0  {\ldots}                   3                  1
204789              1.0  {\ldots}                   1                  1
204790              3.0  {\ldots}                   2                  1
204791              3.0  {\ldots}                   4                  1

        Most Recent Sale  Age Decade  Pure Market Filter  Garage Indicator  \textbackslash{}
0                    1.0        13.2                   0               0.0
1                    1.0         9.6                   1               1.0
2                    0.0        11.2                   1               1.0
3                    1.0         6.3                   1               1.0
4                    0.0         6.3                   1               1.0
{\ldots}                  {\ldots}         {\ldots}                 {\ldots}               {\ldots}
204787               0.0         5.8                   1               1.0
204788               1.0         9.3                   1               1.0
204789               1.0         5.9                   1               1.0
204790               1.0         6.0                   1               1.0
204791               0.0         4.7                   1               0.0

        Neigborhood Code (mapping)  Town and Neighborhood  \textbackslash{}
0                               50                   7650
1                              120                  71120
2                              210                  70210
3                              220                  17220
4                              120                  32120
{\ldots}                            {\ldots}                    {\ldots}
204787                         321                  72321
204788                          21                   2321
204789                          90                   1590
204790                          80                   2280
204791                          30                   7230

                                              Description  Lot Size
0       This property, sold on 09/14/2015, is a one-st{\ldots}    2500.0
1       This property, sold on 05/23/2018, is a one-st{\ldots}    3780.0
2       This property, sold on 02/18/2016, is a one-st{\ldots}    4375.0
3       This property, sold on 07/23/2013, is a one-st{\ldots}    4375.0
4       This property, sold on 06/10/2016, is a one-st{\ldots}    8400.0
{\ldots}                                                   {\ldots}       {\ldots}
204787  This property, sold on 07/23/2014, is a one-st{\ldots}    4375.0
204788  This property, sold on 03/27/2019, is a one-st{\ldots}   16509.0
204789  This property, sold on 01/31/2014, is a one-st{\ldots}    3810.0
204790  This property, sold on 02/22/2018, is a one-st{\ldots}    6650.0
204791  This property, sold on 04/22/2014, is a one-st{\ldots}    2500.0

[204792 rows x 62 columns]
\end{Verbatim}
\end{tcolorbox}
        
    \begin{tcolorbox}[breakable, size=fbox, boxrule=1pt, pad at break*=1mm,colback=cellbackground, colframe=cellborder]
\prompt{In}{incolor}{12}{\boxspacing}
\begin{Verbatim}[commandchars=\\\{\}]
\PY{n}{training\PYZus{}data}\PY{p}{[}\PY{l+s+s1}{\PYZsq{}}\PY{l+s+s1}{Sale Price}\PY{l+s+s1}{\PYZsq{}}\PY{p}{]}
\end{Verbatim}
\end{tcolorbox}

            \begin{tcolorbox}[breakable, size=fbox, boxrule=.5pt, pad at break*=1mm, opacityfill=0]
\prompt{Out}{outcolor}{12}{\boxspacing}
\begin{Verbatim}[commandchars=\\\{\}]
0              1
1         285000
2          22000
3         225000
4          22600
           {\ldots}
204787     37100
204788    225000
204789    135000
204790    392000
204791    125000
Name: Sale Price, Length: 204792, dtype: int64
\end{Verbatim}
\end{tcolorbox}
        
    \begin{tcolorbox}[breakable, size=fbox, boxrule=1pt, pad at break*=1mm,colback=cellbackground, colframe=cellborder]
\prompt{In}{incolor}{13}{\boxspacing}
\begin{Verbatim}[commandchars=\\\{\}]
\PY{n}{training\PYZus{}data} \PY{o}{=} \PY{n}{training\PYZus{}data}\PY{p}{[}\PY{n}{training\PYZus{}data}\PY{p}{[}\PY{l+s+s1}{\PYZsq{}}\PY{l+s+s1}{Sale Price}\PY{l+s+s1}{\PYZsq{}}\PY{p}{]} \PY{o}{\PYZgt{}}\PY{o}{=} \PY{l+m+mi}{500}\PY{p}{]}
\PY{n}{training\PYZus{}data}\PY{p}{[}\PY{l+s+s1}{\PYZsq{}}\PY{l+s+s1}{Log Sale Price}\PY{l+s+s1}{\PYZsq{}}\PY{p}{]} \PY{o}{=} \PY{n}{np}\PY{o}{.}\PY{n}{log}\PY{p}{(}\PY{n}{training\PYZus{}data}\PY{p}{[}\PY{l+s+s1}{\PYZsq{}}\PY{l+s+s1}{Sale Price}\PY{l+s+s1}{\PYZsq{}}\PY{p}{]}\PY{p}{)}
\end{Verbatim}
\end{tcolorbox}

    \begin{tcolorbox}[breakable, size=fbox, boxrule=1pt, pad at break*=1mm,colback=cellbackground, colframe=cellborder]
\prompt{In}{incolor}{14}{\boxspacing}
\begin{Verbatim}[commandchars=\\\{\}]
\PY{n}{grader}\PY{o}{.}\PY{n}{check}\PY{p}{(}\PY{l+s+s2}{\PYZdq{}}\PY{l+s+s2}{q2b}\PY{l+s+s2}{\PYZdq{}}\PY{p}{)}
\end{Verbatim}
\end{tcolorbox}

            \begin{tcolorbox}[breakable, size=fbox, boxrule=.5pt, pad at break*=1mm, opacityfill=0]
\prompt{Out}{outcolor}{14}{\boxspacing}
\begin{Verbatim}[commandchars=\\\{\}]
q2b results: All test cases passed!
\end{Verbatim}
\end{tcolorbox}
        
    Let's create a new distribution plot on the log-transformed sale price.

    \begin{tcolorbox}[breakable, size=fbox, boxrule=1pt, pad at break*=1mm,colback=cellbackground, colframe=cellborder]
\prompt{In}{incolor}{15}{\boxspacing}
\begin{Verbatim}[commandchars=\\\{\}]
\PY{n}{plot\PYZus{}distribution}\PY{p}{(}\PY{n}{training\PYZus{}data}\PY{p}{,} \PY{n}{label}\PY{o}{=}\PY{l+s+s1}{\PYZsq{}}\PY{l+s+s1}{Log Sale Price}\PY{l+s+s1}{\PYZsq{}}\PY{p}{)}\PY{p}{;}
\end{Verbatim}
\end{tcolorbox}

    \begin{center}
    \adjustimage{max size={0.9\linewidth}{0.9\paperheight}}{output_36_0.png}
    \end{center}
    { \hspace*{\fill} \\}
    
    \hypertarget{question-3}{%
\subsection{Question 3}\label{question-3}}

\hypertarget{part-1}{%
\subsubsection{Part 1}\label{part-1}}

To check your understanding of the graph and summary statistics above,
answer the following \texttt{True} or \texttt{False} questions:

\begin{enumerate}
\def\labelenumi{\arabic{enumi}.}
\tightlist
\item
  The distribution of \texttt{Log\ Sale\ Price} in the training set is
  symmetric.
\item
  The mean of \texttt{Log\ Sale\ Price} in the training set is greater
  than the median.
\item
  At least 25\% of the houses in the training set sold for more than
  \$200,000.00.
\end{enumerate}

\emph{The provided tests for this question do not confirm that you have
answered correctly; only that you have assigned each variable to
\texttt{True} or \texttt{False}.}

    \begin{tcolorbox}[breakable, size=fbox, boxrule=1pt, pad at break*=1mm,colback=cellbackground, colframe=cellborder]
\prompt{In}{incolor}{16}{\boxspacing}
\begin{Verbatim}[commandchars=\\\{\}]
\PY{c+c1}{\PYZsh{} These should be True or False}
\PY{n}{q3statement1} \PY{o}{=} \PY{k+kc}{True}
\PY{n}{q3statement2} \PY{o}{=} \PY{k+kc}{False}
\PY{n}{q3statement3} \PY{o}{=} \PY{k+kc}{True}
\end{Verbatim}
\end{tcolorbox}

    \begin{tcolorbox}[breakable, size=fbox, boxrule=1pt, pad at break*=1mm,colback=cellbackground, colframe=cellborder]
\prompt{In}{incolor}{17}{\boxspacing}
\begin{Verbatim}[commandchars=\\\{\}]
\PY{n}{grader}\PY{o}{.}\PY{n}{check}\PY{p}{(}\PY{l+s+s2}{\PYZdq{}}\PY{l+s+s2}{q3a}\PY{l+s+s2}{\PYZdq{}}\PY{p}{)}
\end{Verbatim}
\end{tcolorbox}

            \begin{tcolorbox}[breakable, size=fbox, boxrule=.5pt, pad at break*=1mm, opacityfill=0]
\prompt{Out}{outcolor}{17}{\boxspacing}
\begin{Verbatim}[commandchars=\\\{\}]
q3a results: All test cases passed!
\end{Verbatim}
\end{tcolorbox}
        
    \begin{center}\rule{0.5\linewidth}{0.5pt}\end{center}

\hypertarget{part-2}{%
\subsubsection{Part 2}\label{part-2}}

Next, we want to explore if any there is any correlation between
\texttt{Log\ Sale\ Price} and the total area occupied by the household.
The \texttt{codebook.txt} file tells us the column
\texttt{Building\ Square\ Feet} should do the trick -- it measures
``(from the exterior) the total area, in square feet, occupied by the
building''.

Before creating this jointplot however, let's also apply a log
transformation to the \texttt{Building\ Square\ Feet} column.

In the following cell, create a new column
\texttt{Log\ Building\ Square\ Feet} in our \texttt{training\_data} that
contains the log transformed area occupied by each household.

\textbf{You should NOT remove the original
\texttt{Building\ Square\ Feet} column this time as it will be used for
later questions}.

\emph{To ensure that any errors from this part do not propagate to later
questions, there will be no hidden tests here.}

    \begin{tcolorbox}[breakable, size=fbox, boxrule=1pt, pad at break*=1mm,colback=cellbackground, colframe=cellborder]
\prompt{In}{incolor}{18}{\boxspacing}
\begin{Verbatim}[commandchars=\\\{\}]
\PY{n}{training\PYZus{}data}\PY{p}{[}\PY{l+s+s1}{\PYZsq{}}\PY{l+s+s1}{Log Building Square Feet}\PY{l+s+s1}{\PYZsq{}}\PY{p}{]} \PY{o}{=} \PY{n}{np}\PY{o}{.}\PY{n}{log}\PY{p}{(}\PY{n}{training\PYZus{}data}\PY{p}{[}\PY{l+s+s1}{\PYZsq{}}\PY{l+s+s1}{Building Square Feet}\PY{l+s+s1}{\PYZsq{}}\PY{p}{]}\PY{p}{)}
\end{Verbatim}
\end{tcolorbox}

    \begin{tcolorbox}[breakable, size=fbox, boxrule=1pt, pad at break*=1mm,colback=cellbackground, colframe=cellborder]
\prompt{In}{incolor}{19}{\boxspacing}
\begin{Verbatim}[commandchars=\\\{\}]
\PY{n}{grader}\PY{o}{.}\PY{n}{check}\PY{p}{(}\PY{l+s+s2}{\PYZdq{}}\PY{l+s+s2}{q3b}\PY{l+s+s2}{\PYZdq{}}\PY{p}{)}
\end{Verbatim}
\end{tcolorbox}

            \begin{tcolorbox}[breakable, size=fbox, boxrule=.5pt, pad at break*=1mm, opacityfill=0]
\prompt{Out}{outcolor}{19}{\boxspacing}
\begin{Verbatim}[commandchars=\\\{\}]
q3b results: All test cases passed!
\end{Verbatim}
\end{tcolorbox}
        
    \begin{center}\rule{0.5\linewidth}{0.5pt}\end{center}

\hypertarget{part-3}{%
\subsubsection{Part 3}\label{part-3}}

As shown below, we created a joint plot with
\texttt{Log\ Building\ Square\ Feet} on the x-axis, and
\texttt{Log\ Sale\ Price} on the y-axis. In addition, we fit a simple
linear regression line through the bivariate scatter plot in the middle.

Based on the following plot, does there exist a correlation between
\texttt{Log\ Sale\ Price} and \texttt{Log\ Building\ Square\ Feet}?
Would \texttt{Log\ Building\ Square\ Feet} make a good candidate as one
of the features for our model?

\begin{figure}
\centering
\includegraphics{images/q2p3_jointplot.png}
\caption{Joint Plot}
\end{figure}

    There is a positive correlation between the two variable, however a
linear correlation may not be the most accurate one. I believe Log
Building Square Feet would make a good candidate as one of the feature
for our model as there is a clear correlation between the two variables.

    \hypertarget{question-4}{%
\subsection{Question 4}\label{question-4}}

Continuing from the previous part, as you explore the data set, you
might still run into more outliers that prevent you from creating a
clear visualization or capturing the trend of the majority of the
houses.

For this assignment, we will work to remove these outliers from the data
as we run into them. Write a function \texttt{remove\_outliers} that
removes outliers from a data set based off a threshold value of a
variable. For example,
\texttt{remove\_outliers(training\_data,\ \textquotesingle{}Building\ Square\ Feet\textquotesingle{},\ upper=8000)}
should return a data frame with only observations that satisfy
\texttt{Building\ Square\ Feet} less than or equal to 8000.

\emph{The provided tests check that training\_data was updated
correctly, so that future analyses are not corrupted by a mistake.
However, the provided tests do not check that you have implemented
remove\_outliers correctly so that it works with any data, variable,
lower, and upper bound.}

    \begin{tcolorbox}[breakable, size=fbox, boxrule=1pt, pad at break*=1mm,colback=cellbackground, colframe=cellborder]
\prompt{In}{incolor}{20}{\boxspacing}
\begin{Verbatim}[commandchars=\\\{\}]
\PY{k}{def} \PY{n+nf}{remove\PYZus{}outliers}\PY{p}{(}\PY{n}{data}\PY{p}{,} \PY{n}{variable}\PY{p}{,} \PY{n}{lower}\PY{o}{=}\PY{o}{\PYZhy{}}\PY{n}{np}\PY{o}{.}\PY{n}{inf}\PY{p}{,} \PY{n}{upper}\PY{o}{=}\PY{n}{np}\PY{o}{.}\PY{n}{inf}\PY{p}{)}\PY{p}{:}
\PY{+w}{    }\PY{l+s+sd}{\PYZdq{}\PYZdq{}\PYZdq{}}
\PY{l+s+sd}{    Input:}
\PY{l+s+sd}{      data (data frame): the table to be filtered}
\PY{l+s+sd}{      variable (string): the column with numerical outliers}
\PY{l+s+sd}{      lower (numeric): observations with values lower than this will be removed}
\PY{l+s+sd}{      upper (numeric): observations with values higher than this will be removed}
\PY{l+s+sd}{    }
\PY{l+s+sd}{    Output:}
\PY{l+s+sd}{      a data frame with outliers removed}
\PY{l+s+sd}{      }
\PY{l+s+sd}{    Note: This function should not change mutate the contents of data.}
\PY{l+s+sd}{    \PYZdq{}\PYZdq{}\PYZdq{}}  
    \PY{k}{return} \PY{n}{data}\PY{p}{[}\PY{p}{(}\PY{n}{data}\PY{p}{[}\PY{n}{variable}\PY{p}{]} \PY{o}{\PYZgt{}}\PY{o}{=} \PY{n}{lower}\PY{p}{)} \PY{o}{\PYZam{}} \PY{p}{(}\PY{n}{data}\PY{p}{[}\PY{n}{variable}\PY{p}{]} \PY{o}{\PYZlt{}}\PY{o}{=} \PY{n}{upper}\PY{p}{)}\PY{p}{]}
\end{Verbatim}
\end{tcolorbox}

    \begin{tcolorbox}[breakable, size=fbox, boxrule=1pt, pad at break*=1mm,colback=cellbackground, colframe=cellborder]
\prompt{In}{incolor}{21}{\boxspacing}
\begin{Verbatim}[commandchars=\\\{\}]
\PY{n}{grader}\PY{o}{.}\PY{n}{check}\PY{p}{(}\PY{l+s+s2}{\PYZdq{}}\PY{l+s+s2}{q4}\PY{l+s+s2}{\PYZdq{}}\PY{p}{)}
\end{Verbatim}
\end{tcolorbox}

            \begin{tcolorbox}[breakable, size=fbox, boxrule=.5pt, pad at break*=1mm, opacityfill=0]
\prompt{Out}{outcolor}{21}{\boxspacing}
\begin{Verbatim}[commandchars=\\\{\}]
q4 results: All test cases passed!
\end{Verbatim}
\end{tcolorbox}
        
    \hypertarget{part-3-feature-engineering}{%
\section{Part 3: Feature Engineering}\label{part-3-feature-engineering}}

In this section we will walk you through a few feature engineering
techniques.

\hypertarget{bedrooms}{%
\subsubsection{Bedrooms}\label{bedrooms}}

Let's start simple by extracting the total number of bedrooms as our
first feature for the model. You may notice that the \texttt{Bedrooms}
column doesn't actually exist in the original dataframe! Instead, it is
part of the \texttt{Description} column.

\hypertarget{question-5}{%
\subsection{Question 5}\label{question-5}}

\hypertarget{part-1}{%
\subsubsection{Part 1}\label{part-1}}

Let's take a closer look at the \texttt{Description} column first.
Compare the description across a few rows together at the same time. For
the following list of variables, how many of them can be extracted from
the \texttt{Description} column? Assign your answer as an integer to the
variable \texttt{q4a}. - The date the property was sold on - The number
of stories the property contains - The previous owner of the property -
The address of the property - The number of garages the property has -
The total number of rooms inside the property - The total number of
bedrooms inside the property - The total number of bathrooms inside the
property

    \begin{tcolorbox}[breakable, size=fbox, boxrule=1pt, pad at break*=1mm,colback=cellbackground, colframe=cellborder]
\prompt{In}{incolor}{22}{\boxspacing}
\begin{Verbatim}[commandchars=\\\{\}]
\PY{n+nb}{list}\PY{p}{(}\PY{n}{training\PYZus{}data}\PY{p}{[}\PY{l+s+s2}{\PYZdq{}}\PY{l+s+s2}{Description}\PY{l+s+s2}{\PYZdq{}}\PY{p}{]}\PY{p}{[}\PY{p}{:}\PY{l+m+mi}{5}\PY{p}{]}\PY{p}{)}
\end{Verbatim}
\end{tcolorbox}

            \begin{tcolorbox}[breakable, size=fbox, boxrule=.5pt, pad at break*=1mm, opacityfill=0]
\prompt{Out}{outcolor}{22}{\boxspacing}
\begin{Verbatim}[commandchars=\\\{\}]
['This property, sold on 05/23/2018, is a one-story houeshold located at 2844 N
LOWELL AVE.It has a total of 6 rooms, 3 of which are bedrooms, and 1.0 of which
are bathrooms.',
 'This property, sold on 02/18/2016, is a one-story houeshold located at 11415 S
PRAIRIE AVE.It has a total of 7 rooms, 3 of which are bedrooms, and 1.0 of which
are bathrooms.',
 'This property, sold on 07/23/2013, is a one-story with partially livable
attics houeshold located at 2012 DOBSON ST.It has a total of 5 rooms, 3 of which
are bedrooms, and 1.5 of which are bathrooms.',
 'This property, sold on 06/10/2016, is a one-story houeshold located at 104
SAUK TRL.It has a total of 5 rooms, 2 of which are bedrooms, and 1.0 of which
are bathrooms.',
 'This property, sold on 10/26/2017, is a one-story with partially livable
attics houeshold located at 2820 186TH ST.It has a total of 6 rooms, 4 of which
are bedrooms, and 1.5 of which are bathrooms.']
\end{Verbatim}
\end{tcolorbox}
        
    \begin{tcolorbox}[breakable, size=fbox, boxrule=1pt, pad at break*=1mm,colback=cellbackground, colframe=cellborder]
\prompt{In}{incolor}{23}{\boxspacing}
\begin{Verbatim}[commandchars=\\\{\}]
\PY{n}{q5a} \PY{o}{=} \PY{l+m+mi}{6}
\PY{o}{.}\PY{o}{.}\PY{o}{.}
\end{Verbatim}
\end{tcolorbox}

    \begin{tcolorbox}[breakable, size=fbox, boxrule=1pt, pad at break*=1mm,colback=cellbackground, colframe=cellborder]
\prompt{In}{incolor}{24}{\boxspacing}
\begin{Verbatim}[commandchars=\\\{\}]
\PY{n}{grader}\PY{o}{.}\PY{n}{check}\PY{p}{(}\PY{l+s+s2}{\PYZdq{}}\PY{l+s+s2}{q5a}\PY{l+s+s2}{\PYZdq{}}\PY{p}{)}
\end{Verbatim}
\end{tcolorbox}

            \begin{tcolorbox}[breakable, size=fbox, boxrule=.5pt, pad at break*=1mm, opacityfill=0]
\prompt{Out}{outcolor}{24}{\boxspacing}
\begin{Verbatim}[commandchars=\\\{\}]
q5a results: All test cases passed!
\end{Verbatim}
\end{tcolorbox}
        
    \begin{tcolorbox}[breakable, size=fbox, boxrule=1pt, pad at break*=1mm,colback=cellbackground, colframe=cellborder]
\prompt{In}{incolor}{25}{\boxspacing}
\begin{Verbatim}[commandchars=\\\{\}]
\PY{c+c1}{\PYZsh{} optional cell for scratch work}
\end{Verbatim}
\end{tcolorbox}

    \begin{center}\rule{0.5\linewidth}{0.5pt}\end{center}

\hypertarget{part-2}{%
\subsubsection{Part 2}\label{part-2}}

Write a function \texttt{add\_total\_bedrooms(data)} that returns a copy
of \texttt{data} with an additional column called \texttt{Bedrooms} that
contains the total number of bedrooms (as integers) for each house.
\textbf{Treat missing values as zeros if necessary}. Remember that you
can make use of vectorized code here; you shouldn't need any
\texttt{for} statements.

\textbf{Hint}: You should consider inspecting the \texttt{Description}
column to figure out if there is any general structure within the text.
Once you have noticed a certain pattern, you are set with the power of
Regex!

    \begin{tcolorbox}[breakable, size=fbox, boxrule=1pt, pad at break*=1mm,colback=cellbackground, colframe=cellborder]
\prompt{In}{incolor}{ }{\boxspacing}
\begin{Verbatim}[commandchars=\\\{\}]

\end{Verbatim}
\end{tcolorbox}

    \begin{tcolorbox}[breakable, size=fbox, boxrule=1pt, pad at break*=1mm,colback=cellbackground, colframe=cellborder]
\prompt{In}{incolor}{ }{\boxspacing}
\begin{Verbatim}[commandchars=\\\{\}]
\PY{k}{def} \PY{n+nf}{add\PYZus{}total\PYZus{}bedrooms}\PY{p}{(}\PY{n}{data}\PY{p}{)}\PY{p}{:}
\PY{+w}{    }\PY{l+s+sd}{\PYZdq{}\PYZdq{}\PYZdq{}}
\PY{l+s+sd}{    Input:}
\PY{l+s+sd}{      data (data frame): a data frame containing at least the Description column.}
\PY{l+s+sd}{    \PYZdq{}\PYZdq{}\PYZdq{}}
    \PY{n}{with\PYZus{}rooms} \PY{o}{=} \PY{n}{data}\PY{o}{.}\PY{n}{copy}\PY{p}{(}\PY{p}{)}
    \PY{n}{with\PYZus{}rooms}\PY{p}{[}\PY{l+s+s1}{\PYZsq{}}\PY{l+s+s1}{Bedrooms}\PY{l+s+s1}{\PYZsq{}}\PY{p}{]} \PY{o}{=} \PY{n}{with\PYZus{}rooms}\PY{p}{[}\PY{l+s+s1}{\PYZsq{}}\PY{l+s+s1}{Description}\PY{l+s+s1}{\PYZsq{}}\PY{p}{]}\PY{o}{.}\PY{n}{str}\PY{o}{.}\PY{n}{extract}\PY{p}{(}\PY{l+s+s1}{\PYZsq{}}\PY{l+s+s1}{(}\PY{l+s+s1}{\PYZbs{}}\PY{l+s+s1}{d+) of which are bedrooms}\PY{l+s+s1}{\PYZsq{}}\PY{p}{,} \PY{n}{expand} \PY{o}{=} \PY{k+kc}{True}\PY{p}{)}\PY{o}{.}\PY{n}{fillna}\PY{p}{(}\PY{l+m+mi}{0}\PY{p}{)}\PY{o}{.}\PY{n}{astype}\PY{p}{(}\PY{n+nb}{int}\PY{p}{)}
    \PY{k}{return} \PY{n}{with\PYZus{}rooms}

\PY{n}{training\PYZus{}data} \PY{o}{=} \PY{n}{add\PYZus{}total\PYZus{}bedrooms}\PY{p}{(}\PY{n}{training\PYZus{}data}\PY{p}{)}
\end{Verbatim}
\end{tcolorbox}

    \begin{tcolorbox}[breakable, size=fbox, boxrule=1pt, pad at break*=1mm,colback=cellbackground, colframe=cellborder]
\prompt{In}{incolor}{27}{\boxspacing}
\begin{Verbatim}[commandchars=\\\{\}]
\PY{n}{grader}\PY{o}{.}\PY{n}{check}\PY{p}{(}\PY{l+s+s2}{\PYZdq{}}\PY{l+s+s2}{q5b}\PY{l+s+s2}{\PYZdq{}}\PY{p}{)}
\end{Verbatim}
\end{tcolorbox}

            \begin{tcolorbox}[breakable, size=fbox, boxrule=.5pt, pad at break*=1mm, opacityfill=0]
\prompt{Out}{outcolor}{27}{\boxspacing}
\begin{Verbatim}[commandchars=\\\{\}]
q5b results: All test cases passed!
\end{Verbatim}
\end{tcolorbox}
        
    \begin{center}\rule{0.5\linewidth}{0.5pt}\end{center}

\hypertarget{part-3}{%
\subsubsection{Part 3}\label{part-3}}

Create a visualization that clearly and succintly shows if there exists
an association between \texttt{Bedrooms} and \texttt{Log\ Sale\ Price}.
A good visualization should satisfy the following requirements: - It
should avoid overplotting. - It should have clearly labeled axes and
succinct title. - It should convey the strength of the correlation
between the sale price and the number of rooms.

\textbf{Hint}: A direct scatter plot of the sale price against the
number of rooms for all of the households in our training data might
risk overplotting.

    \begin{tcolorbox}[breakable, size=fbox, boxrule=1pt, pad at break*=1mm,colback=cellbackground, colframe=cellborder]
\prompt{In}{incolor}{28}{\boxspacing}
\begin{Verbatim}[commandchars=\\\{\}]
\PY{n}{sns}\PY{o}{.}\PY{n}{violinplot}\PY{p}{(}\PY{n}{x} \PY{o}{=} \PY{n}{training\PYZus{}data}\PY{p}{[}\PY{l+s+s1}{\PYZsq{}}\PY{l+s+s1}{Bedrooms}\PY{l+s+s1}{\PYZsq{}}\PY{p}{]}\PY{o}{.}\PY{n}{sort\PYZus{}values}\PY{p}{(}\PY{n}{ascending}\PY{o}{=}\PY{k+kc}{True}\PY{p}{)}\PY{p}{,} \PY{n}{y} \PY{o}{=} \PY{n}{training\PYZus{}data}\PY{p}{[}\PY{l+s+s1}{\PYZsq{}}\PY{l+s+s1}{Log Sale Price}\PY{l+s+s1}{\PYZsq{}}\PY{p}{]}\PY{p}{)}
\end{Verbatim}
\end{tcolorbox}

            \begin{tcolorbox}[breakable, size=fbox, boxrule=.5pt, pad at break*=1mm, opacityfill=0]
\prompt{Out}{outcolor}{28}{\boxspacing}
\begin{Verbatim}[commandchars=\\\{\}]
<AxesSubplot:xlabel='Bedrooms', ylabel='Log Sale Price'>
\end{Verbatim}
\end{tcolorbox}
        
    \begin{center}
    \adjustimage{max size={0.9\linewidth}{0.9\paperheight}}{output_58_1.png}
    \end{center}
    { \hspace*{\fill} \\}
    
    \hypertarget{question-6}{%
\subsection{Question 6}\label{question-6}}

    Now, let's take a look at the relationship between neighborhood and sale
prices of the houses in our data set. Notice that currently we don't
have the actual names for the neighborhoods. Instead we will use a
similar column \texttt{Neighborhood\ Code} (which is a numerical
encoding of the actual neighborhoods by the Assessment office).

    \hypertarget{part-1}{%
\subsubsection{Part 1}\label{part-1}}

Before creating any visualization, let's quickly inspect how many
different neighborhoods we are dealing with.

Assign the variable \texttt{num\_neighborhoods} with the total number of
neighborhoods in \texttt{training\_data}.

    \begin{tcolorbox}[breakable, size=fbox, boxrule=1pt, pad at break*=1mm,colback=cellbackground, colframe=cellborder]
\prompt{In}{incolor}{29}{\boxspacing}
\begin{Verbatim}[commandchars=\\\{\}]
\PY{n}{training\PYZus{}data}\PY{p}{[}\PY{l+s+s1}{\PYZsq{}}\PY{l+s+s1}{Neighborhood Code}\PY{l+s+s1}{\PYZsq{}}\PY{p}{]}\PY{o}{.}\PY{n}{unique}
\end{Verbatim}
\end{tcolorbox}

            \begin{tcolorbox}[breakable, size=fbox, boxrule=.5pt, pad at break*=1mm, opacityfill=0]
\prompt{Out}{outcolor}{29}{\boxspacing}
\begin{Verbatim}[commandchars=\\\{\}]
<bound method Series.unique of 1         120
2         210
3         220
4         120
6         181
         {\ldots}
204787    321
204788     21
204789     90
204790     80
204791     30
Name: Neighborhood Code, Length: 168931, dtype: int64>
\end{Verbatim}
\end{tcolorbox}
        
    \begin{tcolorbox}[breakable, size=fbox, boxrule=1pt, pad at break*=1mm,colback=cellbackground, colframe=cellborder]
\prompt{In}{incolor}{30}{\boxspacing}
\begin{Verbatim}[commandchars=\\\{\}]
\PY{n}{num\PYZus{}neighborhoods} \PY{o}{=} \PY{n+nb}{len}\PY{p}{(}\PY{n}{pd}\PY{o}{.}\PY{n}{unique}\PY{p}{(}\PY{n}{training\PYZus{}data}\PY{p}{[}\PY{l+s+s1}{\PYZsq{}}\PY{l+s+s1}{Neighborhood Code}\PY{l+s+s1}{\PYZsq{}}\PY{p}{]}\PY{p}{)}\PY{p}{)}
\PY{n}{num\PYZus{}neighborhoods}
\end{Verbatim}
\end{tcolorbox}

            \begin{tcolorbox}[breakable, size=fbox, boxrule=.5pt, pad at break*=1mm, opacityfill=0]
\prompt{Out}{outcolor}{30}{\boxspacing}
\begin{Verbatim}[commandchars=\\\{\}]
193
\end{Verbatim}
\end{tcolorbox}
        
    \begin{tcolorbox}[breakable, size=fbox, boxrule=1pt, pad at break*=1mm,colback=cellbackground, colframe=cellborder]
\prompt{In}{incolor}{31}{\boxspacing}
\begin{Verbatim}[commandchars=\\\{\}]
\PY{n}{grader}\PY{o}{.}\PY{n}{check}\PY{p}{(}\PY{l+s+s2}{\PYZdq{}}\PY{l+s+s2}{q6a}\PY{l+s+s2}{\PYZdq{}}\PY{p}{)}
\end{Verbatim}
\end{tcolorbox}

            \begin{tcolorbox}[breakable, size=fbox, boxrule=.5pt, pad at break*=1mm, opacityfill=0]
\prompt{Out}{outcolor}{31}{\boxspacing}
\begin{Verbatim}[commandchars=\\\{\}]
q6a results: All test cases passed!
\end{Verbatim}
\end{tcolorbox}
        
    \begin{center}\rule{0.5\linewidth}{0.5pt}\end{center}

\hypertarget{part-2}{%
\subsubsection{Part 2}\label{part-2}}

If we try directly plotting the distribution of
\texttt{Log\ Sale\ Price} for all of the households in each neighborhood
using the \texttt{plot\_categorical} function from the next cell, we
would get the following visualization.
\includegraphics{images/q5p2_catplot.png}

    \begin{tcolorbox}[breakable, size=fbox, boxrule=1pt, pad at break*=1mm,colback=cellbackground, colframe=cellborder]
\prompt{In}{incolor}{32}{\boxspacing}
\begin{Verbatim}[commandchars=\\\{\}]
\PY{k}{def} \PY{n+nf}{plot\PYZus{}categorical}\PY{p}{(}\PY{n}{neighborhoods}\PY{p}{)}\PY{p}{:}
    \PY{n}{fig}\PY{p}{,} \PY{n}{axs} \PY{o}{=} \PY{n}{plt}\PY{o}{.}\PY{n}{subplots}\PY{p}{(}\PY{n}{nrows}\PY{o}{=}\PY{l+m+mi}{2}\PY{p}{)}

    \PY{n}{sns}\PY{o}{.}\PY{n}{boxplot}\PY{p}{(}
        \PY{n}{x}\PY{o}{=}\PY{l+s+s1}{\PYZsq{}}\PY{l+s+s1}{Neighborhood Code}\PY{l+s+s1}{\PYZsq{}}\PY{p}{,}
        \PY{n}{y}\PY{o}{=}\PY{l+s+s1}{\PYZsq{}}\PY{l+s+s1}{Log Sale Price}\PY{l+s+s1}{\PYZsq{}}\PY{p}{,}
        \PY{n}{data}\PY{o}{=}\PY{n}{neighborhoods}\PY{p}{,}
        \PY{n}{ax}\PY{o}{=}\PY{n}{axs}\PY{p}{[}\PY{l+m+mi}{0}\PY{p}{]}\PY{p}{,}
    \PY{p}{)}

    \PY{n}{sns}\PY{o}{.}\PY{n}{countplot}\PY{p}{(}
        \PY{n}{x}\PY{o}{=}\PY{l+s+s1}{\PYZsq{}}\PY{l+s+s1}{Neighborhood Code}\PY{l+s+s1}{\PYZsq{}}\PY{p}{,}
        \PY{n}{data}\PY{o}{=}\PY{n}{neighborhoods}\PY{p}{,} 
        \PY{n}{ax}\PY{o}{=}\PY{n}{axs}\PY{p}{[}\PY{l+m+mi}{1}\PY{p}{]}\PY{p}{,}
    \PY{p}{)}

    \PY{c+c1}{\PYZsh{} Draw median price}
    \PY{n}{axs}\PY{p}{[}\PY{l+m+mi}{0}\PY{p}{]}\PY{o}{.}\PY{n}{axhline}\PY{p}{(}
        \PY{n}{y}\PY{o}{=}\PY{n}{training\PYZus{}data}\PY{p}{[}\PY{l+s+s1}{\PYZsq{}}\PY{l+s+s1}{Log Sale Price}\PY{l+s+s1}{\PYZsq{}}\PY{p}{]}\PY{o}{.}\PY{n}{median}\PY{p}{(}\PY{p}{)}\PY{p}{,} 
        \PY{n}{color}\PY{o}{=}\PY{l+s+s1}{\PYZsq{}}\PY{l+s+s1}{red}\PY{l+s+s1}{\PYZsq{}}\PY{p}{,}
        \PY{n}{linestyle}\PY{o}{=}\PY{l+s+s1}{\PYZsq{}}\PY{l+s+s1}{dotted}\PY{l+s+s1}{\PYZsq{}}
    \PY{p}{)}

    \PY{c+c1}{\PYZsh{} Label the bars with counts}
    \PY{k}{for} \PY{n}{patch} \PY{o+ow}{in} \PY{n}{axs}\PY{p}{[}\PY{l+m+mi}{1}\PY{p}{]}\PY{o}{.}\PY{n}{patches}\PY{p}{:}
        \PY{n}{x} \PY{o}{=} \PY{n}{patch}\PY{o}{.}\PY{n}{get\PYZus{}bbox}\PY{p}{(}\PY{p}{)}\PY{o}{.}\PY{n}{get\PYZus{}points}\PY{p}{(}\PY{p}{)}\PY{p}{[}\PY{p}{:}\PY{p}{,} \PY{l+m+mi}{0}\PY{p}{]}
        \PY{n}{y} \PY{o}{=} \PY{n}{patch}\PY{o}{.}\PY{n}{get\PYZus{}bbox}\PY{p}{(}\PY{p}{)}\PY{o}{.}\PY{n}{get\PYZus{}points}\PY{p}{(}\PY{p}{)}\PY{p}{[}\PY{l+m+mi}{1}\PY{p}{,} \PY{l+m+mi}{1}\PY{p}{]}
        \PY{n}{axs}\PY{p}{[}\PY{l+m+mi}{1}\PY{p}{]}\PY{o}{.}\PY{n}{annotate}\PY{p}{(}\PY{l+s+sa}{f}\PY{l+s+s1}{\PYZsq{}}\PY{l+s+si}{\PYZob{}}\PY{n+nb}{int}\PY{p}{(}\PY{n}{y}\PY{p}{)}\PY{l+s+si}{\PYZcb{}}\PY{l+s+s1}{\PYZsq{}}\PY{p}{,} \PY{p}{(}\PY{n}{x}\PY{o}{.}\PY{n}{mean}\PY{p}{(}\PY{p}{)}\PY{p}{,} \PY{n}{y}\PY{p}{)}\PY{p}{,} \PY{n}{ha}\PY{o}{=}\PY{l+s+s1}{\PYZsq{}}\PY{l+s+s1}{center}\PY{l+s+s1}{\PYZsq{}}\PY{p}{,} \PY{n}{va}\PY{o}{=}\PY{l+s+s1}{\PYZsq{}}\PY{l+s+s1}{bottom}\PY{l+s+s1}{\PYZsq{}}\PY{p}{)}

    \PY{c+c1}{\PYZsh{} Format x\PYZhy{}axes}
    \PY{n}{axs}\PY{p}{[}\PY{l+m+mi}{1}\PY{p}{]}\PY{o}{.}\PY{n}{set\PYZus{}xticklabels}\PY{p}{(}\PY{n}{axs}\PY{p}{[}\PY{l+m+mi}{1}\PY{p}{]}\PY{o}{.}\PY{n}{xaxis}\PY{o}{.}\PY{n}{get\PYZus{}majorticklabels}\PY{p}{(}\PY{p}{)}\PY{p}{,} \PY{n}{rotation}\PY{o}{=}\PY{l+m+mi}{90}\PY{p}{)}
    \PY{n}{axs}\PY{p}{[}\PY{l+m+mi}{0}\PY{p}{]}\PY{o}{.}\PY{n}{xaxis}\PY{o}{.}\PY{n}{set\PYZus{}visible}\PY{p}{(}\PY{k+kc}{False}\PY{p}{)}

    \PY{c+c1}{\PYZsh{} Narrow the gap between the plots}
    \PY{n}{plt}\PY{o}{.}\PY{n}{subplots\PYZus{}adjust}\PY{p}{(}\PY{n}{hspace}\PY{o}{=}\PY{l+m+mf}{0.01}\PY{p}{)}
\end{Verbatim}
\end{tcolorbox}

    Oh no, looks like we have run into the problem of overplotting again!

You might have noticed that the graph is overplotted because
\textbf{there are actually quite a few neighborhoods in our dataset}!
For the clarity of our visualization, we will have to zoom in again on a
few of them. The reason for this is our visualization will become quite
cluttered with a super dense x-axis.

Assign the variable \texttt{in\_top\_20\_neighborhoods} to a copy of
\texttt{training\_data} that contains only top 20 neighborhoods with the
most number of houses.

    \begin{tcolorbox}[breakable, size=fbox, boxrule=1pt, pad at break*=1mm,colback=cellbackground, colframe=cellborder]
\prompt{In}{incolor}{33}{\boxspacing}
\begin{Verbatim}[commandchars=\\\{\}]
\PY{n}{training\PYZus{}data}
\end{Verbatim}
\end{tcolorbox}

            \begin{tcolorbox}[breakable, size=fbox, boxrule=.5pt, pad at break*=1mm, opacityfill=0]
\prompt{Out}{outcolor}{33}{\boxspacing}
\begin{Verbatim}[commandchars=\\\{\}]
                   PIN  Property Class  Neighborhood Code  Land Square Feet  \textbackslash{}
1       13272240180000             202                120            3780.0
2       25221150230000             202                210            4375.0
3       10251130030000             203                220            4375.0
4       31361040550000             202                120            8400.0
6       30314240080000             203                181           10890.0
{\ldots}                {\ldots}             {\ldots}                {\ldots}               {\ldots}
204787  25163010260000             202                321            4375.0
204788   5063010090000             204                 21           16509.0
204789  16333020150000             202                 90            3810.0
204790   9242030500000             203                 80            6650.0
204791  19102030080000             203                 30            2500.0

        Town Code  Apartments  Wall Material  Roof Material  Basement  \textbackslash{}
1              71         0.0            2.0            1.0       1.0
2              70         0.0            2.0            1.0       2.0
3              17         0.0            3.0            1.0       1.0
4              32         0.0            3.0            1.0       2.0
6              37         0.0            1.0            1.0       1.0
{\ldots}           {\ldots}         {\ldots}            {\ldots}            {\ldots}       {\ldots}
204787         72         0.0            2.0            1.0       1.0
204788         23         0.0            1.0            1.0       1.0
204789         15         0.0            2.0            1.0       1.0
204790         22         0.0            2.0            1.0       1.0
204791         72         0.0            1.0            1.0       1.0

        Basement Finish  {\ldots}  Age Decade  Pure Market Filter  \textbackslash{}
1                   1.0  {\ldots}         9.6                   1
2                   3.0  {\ldots}        11.2                   1
3                   3.0  {\ldots}         6.3                   1
4                   3.0  {\ldots}         6.3                   1
6                   3.0  {\ldots}        10.9                   1
{\ldots}                 {\ldots}  {\ldots}         {\ldots}                 {\ldots}
204787              1.0  {\ldots}         5.8                   1
204788              1.0  {\ldots}         9.3                   1
204789              1.0  {\ldots}         5.9                   1
204790              3.0  {\ldots}         6.0                   1
204791              3.0  {\ldots}         4.7                   1

        Garage Indicator  Neigborhood Code (mapping)  Town and Neighborhood  \textbackslash{}
1                    1.0                         120                  71120
2                    1.0                         210                  70210
3                    1.0                         220                  17220
4                    1.0                         120                  32120
6                    1.0                         181                  37181
{\ldots}                  {\ldots}                         {\ldots}                    {\ldots}
204787               1.0                         321                  72321
204788               1.0                          21                   2321
204789               1.0                          90                   1590
204790               1.0                          80                   2280
204791               0.0                          30                   7230

                                              Description  Lot Size  \textbackslash{}
1       This property, sold on 05/23/2018, is a one-st{\ldots}    3780.0
2       This property, sold on 02/18/2016, is a one-st{\ldots}    4375.0
3       This property, sold on 07/23/2013, is a one-st{\ldots}    4375.0
4       This property, sold on 06/10/2016, is a one-st{\ldots}    8400.0
6       This property, sold on 10/26/2017, is a one-st{\ldots}   10890.0
{\ldots}                                                   {\ldots}       {\ldots}
204787  This property, sold on 07/23/2014, is a one-st{\ldots}    4375.0
204788  This property, sold on 03/27/2019, is a one-st{\ldots}   16509.0
204789  This property, sold on 01/31/2014, is a one-st{\ldots}    3810.0
204790  This property, sold on 02/22/2018, is a one-st{\ldots}    6650.0
204791  This property, sold on 04/22/2014, is a one-st{\ldots}    2500.0

        Log Sale Price  Log Building Square Feet  Bedrooms
1            12.560244                  6.904751         3
2             9.998798                  6.810142         3
3            12.323856                  7.068172         3
4            10.025705                  6.855409         2
6            11.512925                  7.458186         4
{\ldots}                {\ldots}                       {\ldots}       {\ldots}
204787       10.521372                  6.813445         2
204788       12.323856                  7.603399         4
204789       11.813030                  6.815640         3
204790       12.879017                  7.092574         3
204791       11.736069                  6.946976         2

[168931 rows x 65 columns]
\end{Verbatim}
\end{tcolorbox}
        
    \begin{tcolorbox}[breakable, size=fbox, boxrule=1pt, pad at break*=1mm,colback=cellbackground, colframe=cellborder]
\prompt{In}{incolor}{34}{\boxspacing}
\begin{Verbatim}[commandchars=\\\{\}]
\PY{n}{top\PYZus{}neigborhoods} \PY{o}{=} \PY{n}{training\PYZus{}data}\PY{o}{.}\PY{n}{groupby}\PY{p}{(}\PY{l+s+s1}{\PYZsq{}}\PY{l+s+s1}{Neighborhood Code}\PY{l+s+s1}{\PYZsq{}}\PY{p}{)}\PY{p}{[}\PY{l+s+s1}{\PYZsq{}}\PY{l+s+s1}{Neighborhood Code}\PY{l+s+s1}{\PYZsq{}}\PY{p}{]}\PY{o}{.}\PY{n}{agg}\PY{p}{(}\PY{l+s+s1}{\PYZsq{}}\PY{l+s+s1}{count}\PY{l+s+s1}{\PYZsq{}}\PY{p}{)}\PY{o}{.}\PY{n}{sort\PYZus{}values}\PY{p}{(}\PY{n}{ascending} \PY{o}{=} \PY{k+kc}{False}\PY{p}{)}\PY{o}{.}\PY{n}{head}\PY{p}{(}\PY{l+m+mi}{20}\PY{p}{)}
\PY{n}{in\PYZus{}top\PYZus{}20\PYZus{}neighborhoods} \PY{o}{=} \PY{n}{training\PYZus{}data}\PY{p}{[}\PY{n}{training\PYZus{}data}\PY{p}{[}\PY{l+s+s1}{\PYZsq{}}\PY{l+s+s1}{Neighborhood Code}\PY{l+s+s1}{\PYZsq{}}\PY{p}{]}\PY{o}{.}\PY{n}{isin}\PY{p}{(}\PY{n}{top\PYZus{}neigborhoods}\PY{o}{.}\PY{n}{index}\PY{p}{)}\PY{p}{]}
\PY{n}{in\PYZus{}top\PYZus{}20\PYZus{}neighborhoods}
\end{Verbatim}
\end{tcolorbox}

            \begin{tcolorbox}[breakable, size=fbox, boxrule=.5pt, pad at break*=1mm, opacityfill=0]
\prompt{Out}{outcolor}{34}{\boxspacing}
\begin{Verbatim}[commandchars=\\\{\}]
                   PIN  Property Class  Neighborhood Code  Land Square Feet  \textbackslash{}
1       13272240180000             202                120            3780.0
4       31361040550000             202                120            8400.0
8       13232040260000             205                 70            3100.0
10      19074270080000             202                380            3750.0
11      15083050330000             203                 20            5092.0
{\ldots}                {\ldots}             {\ldots}                {\ldots}               {\ldots}
204781  20361190390000             203                 80            4405.0
204785   9284030280000             202                 40            6650.0
204786   8141120110000             203                100           10010.0
204790   9242030500000             203                 80            6650.0
204791  19102030080000             203                 30            2500.0

        Town Code  Apartments  Wall Material  Roof Material  Basement  \textbackslash{}
1              71         0.0            2.0            1.0       1.0
4              32         0.0            3.0            1.0       2.0
8              71         0.0            2.0            2.0       1.0
10             72         0.0            1.0            1.0       2.0
11             31         0.0            2.0            1.0       1.0
{\ldots}           {\ldots}         {\ldots}            {\ldots}            {\ldots}       {\ldots}
204781         70         0.0            2.0            1.0       1.0
204785         22         0.0            1.0            1.0       1.0
204786         16         0.0            2.0            1.0       1.0
204790         22         0.0            2.0            1.0       1.0
204791         72         0.0            1.0            1.0       1.0

        Basement Finish  {\ldots}  Age Decade  Pure Market Filter  \textbackslash{}
1                   1.0  {\ldots}         9.6                   1
4                   3.0  {\ldots}         6.3                   1
8                   3.0  {\ldots}        10.0                   1
10                  3.0  {\ldots}         7.4                   1
11                  1.0  {\ldots}         5.8                   1
{\ldots}                 {\ldots}  {\ldots}         {\ldots}                 {\ldots}
204781              3.0  {\ldots}         5.7                   1
204785              3.0  {\ldots}         6.1                   1
204786              1.0  {\ldots}         5.6                   1
204790              3.0  {\ldots}         6.0                   1
204791              3.0  {\ldots}         4.7                   1

        Garage Indicator  Neigborhood Code (mapping)  Town and Neighborhood  \textbackslash{}
1                    1.0                         120                  71120
4                    1.0                         120                  32120
8                    1.0                          70                   7170
10                   1.0                         380                  72380
11                   1.0                          20                   3120
{\ldots}                  {\ldots}                         {\ldots}                    {\ldots}
204781               1.0                          80                   7080
204785               1.0                          40                   2240
204786               1.0                         100                  16100
204790               1.0                          80                   2280
204791               0.0                          30                   7230

                                              Description  Lot Size  \textbackslash{}
1       This property, sold on 05/23/2018, is a one-st{\ldots}    3780.0
4       This property, sold on 06/10/2016, is a one-st{\ldots}    8400.0
8       This property, sold on 08/25/2016, is a two-st{\ldots}    3100.0
10      This property, sold on 05/01/2017, is a one-st{\ldots}    3750.0
11      This property, sold on 04/29/2014, is a one-st{\ldots}    5092.0
{\ldots}                                                   {\ldots}       {\ldots}
204781  This property, sold on 07/15/2013, is a one-st{\ldots}    4405.0
204785  This property, sold on 04/03/2014, is a one-st{\ldots}    6650.0
204786  This property, sold on 09/08/2016, is a one-st{\ldots}   10010.0
204790  This property, sold on 02/22/2018, is a one-st{\ldots}    6650.0
204791  This property, sold on 04/22/2014, is a one-st{\ldots}    2500.0

        Log Sale Price  Log Building Square Feet  Bedrooms
1            12.560244                  6.904751         3
4            10.025705                  6.855409         2
8            13.422468                  7.636270         4
10           11.695247                  6.841615         2
11           11.184421                  6.911747         3
{\ldots}                {\ldots}                       {\ldots}       {\ldots}
204781       10.913269                  7.141245         3
204785       11.736069                  6.761573         3
204786       12.568978                  6.948897         3
204790       12.879017                  7.092574         3
204791       11.736069                  6.946976         2

[85526 rows x 65 columns]
\end{Verbatim}
\end{tcolorbox}
        
    \begin{tcolorbox}[breakable, size=fbox, boxrule=1pt, pad at break*=1mm,colback=cellbackground, colframe=cellborder]
\prompt{In}{incolor}{35}{\boxspacing}
\begin{Verbatim}[commandchars=\\\{\}]
\PY{n}{grader}\PY{o}{.}\PY{n}{check}\PY{p}{(}\PY{l+s+s2}{\PYZdq{}}\PY{l+s+s2}{q6b}\PY{l+s+s2}{\PYZdq{}}\PY{p}{)}
\end{Verbatim}
\end{tcolorbox}

            \begin{tcolorbox}[breakable, size=fbox, boxrule=.5pt, pad at break*=1mm, opacityfill=0]
\prompt{Out}{outcolor}{35}{\boxspacing}
\begin{Verbatim}[commandchars=\\\{\}]
q6b results: All test cases passed!
\end{Verbatim}
\end{tcolorbox}
        
    Let's create another of the distribution of sale price within in each
neighborhood again, but this time with a narrower focus!

    \begin{tcolorbox}[breakable, size=fbox, boxrule=1pt, pad at break*=1mm,colback=cellbackground, colframe=cellborder]
\prompt{In}{incolor}{36}{\boxspacing}
\begin{Verbatim}[commandchars=\\\{\}]
\PY{n}{plot\PYZus{}categorical}\PY{p}{(}\PY{n}{neighborhoods}\PY{o}{=}\PY{n}{in\PYZus{}top\PYZus{}20\PYZus{}neighborhoods}\PY{p}{)}
\end{Verbatim}
\end{tcolorbox}

    \begin{center}
    \adjustimage{max size={0.9\linewidth}{0.9\paperheight}}{output_72_0.png}
    \end{center}
    { \hspace*{\fill} \\}
    
    \begin{center}\rule{0.5\linewidth}{0.5pt}\end{center}

\hypertarget{part-3}{%
\subsubsection{Part 3}\label{part-3}}

It looks a lot better now than before, right? Based on the plot above,
what can be said about the relationship between the houses'
\texttt{Log\ Sale\ Price} and their neighborhoods?

    There is no visible relationship between the neighborhoods and the
houses' log sale price. For example, the neighborhoods with the highest
count such as the neighborhood with code 30 has has a lower median than
the overall meadian of all the houses while the neighborhood with code
10 has a higher median than the overall median and it has the third
highest count. Therefore, there is no distinct relationship between the
neighborhood codes and the log scale prices.

    \begin{center}\rule{0.5\linewidth}{0.5pt}\end{center}

\hypertarget{part-4}{%
\subsubsection{Part 4}\label{part-4}}

One way we can deal with the lack of data from some neighborhoods is to
create a new feature that bins neighborhoods together. Let's categorize
our neighborhoods in a crude way: we'll take the top 3 neighborhoods
measured by median \texttt{Log\ Sale\ Price} and identify them as
``expensive neighborhoods''; the other neighborhoods are not marked.

Write a function that returns list of the neighborhood codes of the top
\texttt{n} most pricy neighborhoods as measured by our choice of
aggregating function. For example, in the setup above, we would want to
call
\texttt{find\_expensive\_neighborhoods(training\_data,\ 3,\ np.median)}
to find the top 3 neighborhoods measured by median
\texttt{Log\ Sale\ Price}.

    \begin{tcolorbox}[breakable, size=fbox, boxrule=1pt, pad at break*=1mm,colback=cellbackground, colframe=cellborder]
\prompt{In}{incolor}{37}{\boxspacing}
\begin{Verbatim}[commandchars=\\\{\}]
\PY{k}{def} \PY{n+nf}{find\PYZus{}expensive\PYZus{}neighborhoods}\PY{p}{(}\PY{n}{data}\PY{p}{,} \PY{n}{n}\PY{o}{=}\PY{l+m+mi}{3}\PY{p}{,} \PY{n}{metric}\PY{o}{=}\PY{n}{np}\PY{o}{.}\PY{n}{median}\PY{p}{)}\PY{p}{:}
\PY{+w}{    }\PY{l+s+sd}{\PYZdq{}\PYZdq{}\PYZdq{}}
\PY{l+s+sd}{    Input:}
\PY{l+s+sd}{      data (data frame): should contain at least a string\PYZhy{}valued \PYZsq{}Neighborhood Code\PYZsq{}}
\PY{l+s+sd}{        and a numeric \PYZsq{}Sale Price\PYZsq{} column}
\PY{l+s+sd}{      n (int): the number of top values desired}
\PY{l+s+sd}{      metric (function): function used for aggregating the data in each neighborhood.}
\PY{l+s+sd}{        for example, np.median for median prices}
\PY{l+s+sd}{    }
\PY{l+s+sd}{    Output:}
\PY{l+s+sd}{      a list of the the neighborhood codes of the top n highest\PYZhy{}priced neighborhoods as measured by the metric function}
\PY{l+s+sd}{    \PYZdq{}\PYZdq{}\PYZdq{}}
    \PY{n}{neighborhoods} \PY{o}{=} \PY{n}{data}\PY{o}{.}\PY{n}{groupby}\PY{p}{(}\PY{n}{data}\PY{p}{[}\PY{l+s+s1}{\PYZsq{}}\PY{l+s+s1}{Neighborhood Code}\PY{l+s+s1}{\PYZsq{}}\PY{p}{]}\PY{p}{)}\PY{o}{.}\PY{n}{agg}\PY{p}{(}\PY{n}{metric}\PY{p}{)}\PY{o}{.}\PY{n}{nlargest}\PY{p}{(}\PY{n}{n}\PY{p}{,}\PY{l+s+s1}{\PYZsq{}}\PY{l+s+s1}{Log Sale Price}\PY{l+s+s1}{\PYZsq{}}\PY{p}{)}\PY{o}{.}\PY{n}{index}\PY{o}{.}\PY{n}{to\PYZus{}list}\PY{p}{(}\PY{p}{)}
    
    \PY{c+c1}{\PYZsh{} This makes sure the final list contains the generic int type used in Python3, not specific ones used in numpy.}
    \PY{k}{return} \PY{p}{[}\PY{n+nb}{int}\PY{p}{(}\PY{n}{code}\PY{p}{)} \PY{k}{for} \PY{n}{code} \PY{o+ow}{in} \PY{n}{neighborhoods}\PY{p}{]}

\PY{n}{expensive\PYZus{}neighborhoods} \PY{o}{=} \PY{n}{find\PYZus{}expensive\PYZus{}neighborhoods}\PY{p}{(}\PY{n}{training\PYZus{}data}\PY{p}{,} \PY{l+m+mi}{3}\PY{p}{,} \PY{n}{np}\PY{o}{.}\PY{n}{median}\PY{p}{)}
\PY{n}{expensive\PYZus{}neighborhoods}
\end{Verbatim}
\end{tcolorbox}

            \begin{tcolorbox}[breakable, size=fbox, boxrule=.5pt, pad at break*=1mm, opacityfill=0]
\prompt{Out}{outcolor}{37}{\boxspacing}
\begin{Verbatim}[commandchars=\\\{\}]
[44, 94, 93]
\end{Verbatim}
\end{tcolorbox}
        
    \begin{tcolorbox}[breakable, size=fbox, boxrule=1pt, pad at break*=1mm,colback=cellbackground, colframe=cellborder]
\prompt{In}{incolor}{38}{\boxspacing}
\begin{Verbatim}[commandchars=\\\{\}]
\PY{n}{grader}\PY{o}{.}\PY{n}{check}\PY{p}{(}\PY{l+s+s2}{\PYZdq{}}\PY{l+s+s2}{q6d}\PY{l+s+s2}{\PYZdq{}}\PY{p}{)}
\end{Verbatim}
\end{tcolorbox}

            \begin{tcolorbox}[breakable, size=fbox, boxrule=.5pt, pad at break*=1mm, opacityfill=0]
\prompt{Out}{outcolor}{38}{\boxspacing}
\begin{Verbatim}[commandchars=\\\{\}]
q6d results: All test cases passed!
\end{Verbatim}
\end{tcolorbox}
        
    \begin{center}\rule{0.5\linewidth}{0.5pt}\end{center}

\hypertarget{part-5}{%
\subsubsection{Part 5}\label{part-5}}

We now have a list of neighborhoods we've deemed as higher-priced than
others. Let's use that information to write a function
\texttt{add\_expensive\_neighborhood} that adds a column
\texttt{in\_expensive\_neighborhood} which takes on the value 1 if the
house is part of \texttt{expensive\_neighborhoods} and the value 0
otherwise. This type of variable is known as an \textbf{indicator
variable}.

\textbf{Hint:}
\href{https://pandas.pydata.org/pandas-docs/version/0.23.4/generated/pandas.Series.astype.html}{\texttt{pd.Series.astype}}
may be useful for converting True/False values to integers.

    \begin{tcolorbox}[breakable, size=fbox, boxrule=1pt, pad at break*=1mm,colback=cellbackground, colframe=cellborder]
\prompt{In}{incolor}{39}{\boxspacing}
\begin{Verbatim}[commandchars=\\\{\}]
\PY{k}{def} \PY{n+nf}{add\PYZus{}in\PYZus{}expensive\PYZus{}neighborhood}\PY{p}{(}\PY{n}{data}\PY{p}{,} \PY{n}{neighborhoods}\PY{p}{)}\PY{p}{:}
\PY{+w}{    }\PY{l+s+sd}{\PYZdq{}\PYZdq{}\PYZdq{}}
\PY{l+s+sd}{    Input:}
\PY{l+s+sd}{      data (data frame): a data frame containing a \PYZsq{}Neighborhood Code\PYZsq{} column with values}
\PY{l+s+sd}{        found in the codebook}
\PY{l+s+sd}{      neighborhoods (list of strings): strings should be the names of neighborhoods}
\PY{l+s+sd}{        pre\PYZhy{}identified as expensive}
\PY{l+s+sd}{    Output:}
\PY{l+s+sd}{      data frame identical to the input with the addition of a binary}
\PY{l+s+sd}{      in\PYZus{}expensive\PYZus{}neighborhood column}
\PY{l+s+sd}{    \PYZdq{}\PYZdq{}\PYZdq{}}
    \PY{n}{data}\PY{p}{[}\PY{l+s+s1}{\PYZsq{}}\PY{l+s+s1}{in\PYZus{}expensive\PYZus{}neighborhood}\PY{l+s+s1}{\PYZsq{}}\PY{p}{]} \PY{o}{=} \PY{p}{(}\PY{n}{data}\PY{p}{[}\PY{l+s+s1}{\PYZsq{}}\PY{l+s+s1}{Neighborhood Code}\PY{l+s+s1}{\PYZsq{}}\PY{p}{]}\PY{o}{.}\PY{n}{isin}\PY{p}{(}\PY{n}{neighborhoods}\PY{p}{)}\PY{p}{)}\PY{o}{.}\PY{n}{astype}\PY{p}{(}\PY{n+nb}{int}\PY{p}{)}
    \PY{k}{return} \PY{n}{data}

\PY{n}{expensive\PYZus{}neighborhoods} \PY{o}{=} \PY{n}{find\PYZus{}expensive\PYZus{}neighborhoods}\PY{p}{(}\PY{n}{training\PYZus{}data}\PY{p}{,} \PY{l+m+mi}{3}\PY{p}{,} \PY{n}{np}\PY{o}{.}\PY{n}{median}\PY{p}{)}
\PY{n}{training\PYZus{}data} \PY{o}{=} \PY{n}{add\PYZus{}in\PYZus{}expensive\PYZus{}neighborhood}\PY{p}{(}\PY{n}{training\PYZus{}data}\PY{p}{,} \PY{n}{expensive\PYZus{}neighborhoods}\PY{p}{)}
\end{Verbatim}
\end{tcolorbox}

    \begin{tcolorbox}[breakable, size=fbox, boxrule=1pt, pad at break*=1mm,colback=cellbackground, colframe=cellborder]
\prompt{In}{incolor}{40}{\boxspacing}
\begin{Verbatim}[commandchars=\\\{\}]
\PY{n}{grader}\PY{o}{.}\PY{n}{check}\PY{p}{(}\PY{l+s+s2}{\PYZdq{}}\PY{l+s+s2}{q6e}\PY{l+s+s2}{\PYZdq{}}\PY{p}{)}
\end{Verbatim}
\end{tcolorbox}

            \begin{tcolorbox}[breakable, size=fbox, boxrule=.5pt, pad at break*=1mm, opacityfill=0]
\prompt{Out}{outcolor}{40}{\boxspacing}
\begin{Verbatim}[commandchars=\\\{\}]
q6e results: All test cases passed!
\end{Verbatim}
\end{tcolorbox}
        
    \hypertarget{question-7}{%
\subsection{Question 7}\label{question-7}}

In the following question, we will take a closer look at the
\texttt{Roof\ Material} feature of the dataset and examine how we can
incorporate categorical features into our linear model.

    \hypertarget{part-1}{%
\subsubsection{Part 1}\label{part-1}}

If we look at \texttt{codebook.txt} carefully, we can see that the
Assessor's Office uses the following mapping for the numerical values in
the \texttt{Roof\ Material} column.

\begin{verbatim}
Central Heating (Nominal): 

       1    Shingle/Asphalt
       2    Tar&Gravel
       3    Slate
       4    Shake
       5    Tile
       6    Other
\end{verbatim}

Write a function \texttt{substitute\_roof\_material} that replaces each
numerical value in \texttt{Roof\ Material} with their corresponding roof
material. Your function should return a new DataFrame, not modify the
existing DataFrame.

\textbf{Hint}: the
\href{https://pandas.pydata.org/pandas-docs/stable/generated/pandas.DataFrame.replace.html}{DataFrame.replace}
method may be useful here.

    \begin{tcolorbox}[breakable, size=fbox, boxrule=1pt, pad at break*=1mm,colback=cellbackground, colframe=cellborder]
\prompt{In}{incolor}{41}{\boxspacing}
\begin{Verbatim}[commandchars=\\\{\}]
\PY{k}{def} \PY{n+nf}{substitute\PYZus{}roof\PYZus{}material}\PY{p}{(}\PY{n}{data}\PY{p}{)}\PY{p}{:}
\PY{+w}{    }\PY{l+s+sd}{\PYZdq{}\PYZdq{}\PYZdq{}}
\PY{l+s+sd}{    Input:}
\PY{l+s+sd}{      data (data frame): a data frame containing a \PYZsq{}Roof Material\PYZsq{} column.  Its values}
\PY{l+s+sd}{                         should be limited to those found in the codebook}
\PY{l+s+sd}{    Output:}
\PY{l+s+sd}{      data frame identical to the input except with a refactored \PYZsq{}Roof Material\PYZsq{} column}
\PY{l+s+sd}{    \PYZdq{}\PYZdq{}\PYZdq{}}
    \PY{n}{data2} \PY{o}{=} \PY{n}{data}\PY{o}{.}\PY{n}{copy}\PY{p}{(}\PY{p}{)}
    \PY{n}{data2}  \PY{o}{=} \PY{n}{data2}\PY{o}{.}\PY{n}{replace}\PY{p}{(}\PY{p}{\PYZob{}}\PY{l+s+s1}{\PYZsq{}}\PY{l+s+s1}{Roof Material}\PY{l+s+s1}{\PYZsq{}}\PY{p}{:} \PY{p}{\PYZob{}}\PY{l+m+mi}{1} \PY{p}{:} \PY{l+s+s1}{\PYZsq{}}\PY{l+s+s1}{Shingle/Asphalt}\PY{l+s+s1}{\PYZsq{}}\PY{p}{,} \PY{l+m+mi}{2}\PY{p}{:}\PY{l+s+s1}{\PYZsq{}}\PY{l+s+s1}{Tar\PYZam{}Gravel}\PY{l+s+s1}{\PYZsq{}}\PY{p}{,} \PY{l+m+mi}{3}\PY{p}{:} \PY{l+s+s1}{\PYZsq{}}\PY{l+s+s1}{Slate}\PY{l+s+s1}{\PYZsq{}}\PY{p}{,} \PY{l+m+mi}{4}\PY{p}{:} \PY{l+s+s1}{\PYZsq{}}\PY{l+s+s1}{Shake}\PY{l+s+s1}{\PYZsq{}}\PY{p}{,} \PY{l+m+mi}{5}\PY{p}{:} \PY{l+s+s1}{\PYZsq{}}\PY{l+s+s1}{Tile}\PY{l+s+s1}{\PYZsq{}}\PY{p}{,} \PY{l+m+mi}{6}\PY{p}{:} \PY{l+s+s1}{\PYZsq{}}\PY{l+s+s1}{Other}\PY{l+s+s1}{\PYZsq{}}\PY{p}{\PYZcb{}}\PY{p}{\PYZcb{}}\PY{p}{)}
    \PY{k}{return} \PY{n}{data2}
    
\PY{n}{training\PYZus{}data} \PY{o}{=} \PY{n}{substitute\PYZus{}roof\PYZus{}material}\PY{p}{(}\PY{n}{training\PYZus{}data}\PY{p}{)}
\PY{n}{training\PYZus{}data}\PY{o}{.}\PY{n}{head}\PY{p}{(}\PY{p}{)}
\PY{n}{training\PYZus{}data}
\end{Verbatim}
\end{tcolorbox}

            \begin{tcolorbox}[breakable, size=fbox, boxrule=.5pt, pad at break*=1mm, opacityfill=0]
\prompt{Out}{outcolor}{41}{\boxspacing}
\begin{Verbatim}[commandchars=\\\{\}]
                   PIN  Property Class  Neighborhood Code  Land Square Feet  \textbackslash{}
1       13272240180000             202                120            3780.0
2       25221150230000             202                210            4375.0
3       10251130030000             203                220            4375.0
4       31361040550000             202                120            8400.0
6       30314240080000             203                181           10890.0
{\ldots}                {\ldots}             {\ldots}                {\ldots}               {\ldots}
204787  25163010260000             202                321            4375.0
204788   5063010090000             204                 21           16509.0
204789  16333020150000             202                 90            3810.0
204790   9242030500000             203                 80            6650.0
204791  19102030080000             203                 30            2500.0

        Town Code  Apartments  Wall Material    Roof Material  Basement  \textbackslash{}
1              71         0.0            2.0  Shingle/Asphalt       1.0
2              70         0.0            2.0  Shingle/Asphalt       2.0
3              17         0.0            3.0  Shingle/Asphalt       1.0
4              32         0.0            3.0  Shingle/Asphalt       2.0
6              37         0.0            1.0  Shingle/Asphalt       1.0
{\ldots}           {\ldots}         {\ldots}            {\ldots}              {\ldots}       {\ldots}
204787         72         0.0            2.0  Shingle/Asphalt       1.0
204788         23         0.0            1.0  Shingle/Asphalt       1.0
204789         15         0.0            2.0  Shingle/Asphalt       1.0
204790         22         0.0            2.0  Shingle/Asphalt       1.0
204791         72         0.0            1.0  Shingle/Asphalt       1.0

        Basement Finish  {\ldots}  Pure Market Filter  Garage Indicator  \textbackslash{}
1                   1.0  {\ldots}                   1               1.0
2                   3.0  {\ldots}                   1               1.0
3                   3.0  {\ldots}                   1               1.0
4                   3.0  {\ldots}                   1               1.0
6                   3.0  {\ldots}                   1               1.0
{\ldots}                 {\ldots}  {\ldots}                 {\ldots}               {\ldots}
204787              1.0  {\ldots}                   1               1.0
204788              1.0  {\ldots}                   1               1.0
204789              1.0  {\ldots}                   1               1.0
204790              3.0  {\ldots}                   1               1.0
204791              3.0  {\ldots}                   1               0.0

        Neigborhood Code (mapping)  Town and Neighborhood  \textbackslash{}
1                              120                  71120
2                              210                  70210
3                              220                  17220
4                              120                  32120
6                              181                  37181
{\ldots}                            {\ldots}                    {\ldots}
204787                         321                  72321
204788                          21                   2321
204789                          90                   1590
204790                          80                   2280
204791                          30                   7230

                                              Description  Lot Size  \textbackslash{}
1       This property, sold on 05/23/2018, is a one-st{\ldots}    3780.0
2       This property, sold on 02/18/2016, is a one-st{\ldots}    4375.0
3       This property, sold on 07/23/2013, is a one-st{\ldots}    4375.0
4       This property, sold on 06/10/2016, is a one-st{\ldots}    8400.0
6       This property, sold on 10/26/2017, is a one-st{\ldots}   10890.0
{\ldots}                                                   {\ldots}       {\ldots}
204787  This property, sold on 07/23/2014, is a one-st{\ldots}    4375.0
204788  This property, sold on 03/27/2019, is a one-st{\ldots}   16509.0
204789  This property, sold on 01/31/2014, is a one-st{\ldots}    3810.0
204790  This property, sold on 02/22/2018, is a one-st{\ldots}    6650.0
204791  This property, sold on 04/22/2014, is a one-st{\ldots}    2500.0

        Log Sale Price  Log Building Square Feet  Bedrooms  \textbackslash{}
1            12.560244                  6.904751         3
2             9.998798                  6.810142         3
3            12.323856                  7.068172         3
4            10.025705                  6.855409         2
6            11.512925                  7.458186         4
{\ldots}                {\ldots}                       {\ldots}       {\ldots}
204787       10.521372                  6.813445         2
204788       12.323856                  7.603399         4
204789       11.813030                  6.815640         3
204790       12.879017                  7.092574         3
204791       11.736069                  6.946976         2

        in\_expensive\_neighborhood
1                               0
2                               0
3                               0
4                               0
6                               0
{\ldots}                           {\ldots}
204787                          0
204788                          0
204789                          0
204790                          0
204791                          0

[168931 rows x 66 columns]
\end{Verbatim}
\end{tcolorbox}
        
    \begin{tcolorbox}[breakable, size=fbox, boxrule=1pt, pad at break*=1mm,colback=cellbackground, colframe=cellborder]
\prompt{In}{incolor}{42}{\boxspacing}
\begin{Verbatim}[commandchars=\\\{\}]
\PY{n}{grader}\PY{o}{.}\PY{n}{check}\PY{p}{(}\PY{l+s+s2}{\PYZdq{}}\PY{l+s+s2}{q7a}\PY{l+s+s2}{\PYZdq{}}\PY{p}{)}
\end{Verbatim}
\end{tcolorbox}

            \begin{tcolorbox}[breakable, size=fbox, boxrule=.5pt, pad at break*=1mm, opacityfill=0]
\prompt{Out}{outcolor}{42}{\boxspacing}
\begin{Verbatim}[commandchars=\\\{\}]
q7a results: All test cases passed!
\end{Verbatim}
\end{tcolorbox}
        
    \begin{center}\rule{0.5\linewidth}{0.5pt}\end{center}

\hypertarget{part-2}{%
\subsubsection{Part 2}\label{part-2}}

\hypertarget{an-important-note-on-one-hot-encoding}{%
\paragraph{An Important Note on One Hot
Encoding}\label{an-important-note-on-one-hot-encoding}}

Unfortunately, simply fixing these missing values isn't sufficient for
using \texttt{Roof\ Material} in our model. Since
\texttt{Roof\ Material} is a categorical variable, we will have to
one-hot-encode the data. Notice in the example code below that we have
to pre-specify the categories. For more information on categorical data
in pandas, refer to this
\href{https://pandas-docs.github.io/pandas-docs-travis/user_guide/categorical.html}{link}.
For more information on why we want to use one-hot-encoding, refer to
this
\href{https://machinelearningmastery.com/why-one-hot-encode-data-in-machine-learning/}{link}.

Complete the following function \texttt{ohe\_roof\_material} that
returns a dataframe with the new column one-hot-encoded on the roof
material of the household. These new columns should have the form
\texttt{Roof\ Material\_MATERIAL}. Your function should return a new
DataFrame, not modify the existing DataFrame.

\textbf{Note}: You should \textbf{avoid using \texttt{pd.get\_dummies}}
in your solution as it will remove your original column and is therefore
not as reusable as your constructed data preprocessing pipeline.
Instead, you can one-hot-encode one column into multiple columns
\textbf{using Scikit-learn's
\href{https://scikit-learn.org/stable/modules/generated/sklearn.preprocessing.OneHotEncoder.html}{One
Hot Encoder}}. It's far more customizable!

\emph{Hint}: To get you started with this subpart, here is code that
initializes a \texttt{OneHotEncoding} pre-processing ``model'' from
Scikit-learn and fits it on a simple dataset containing (some of) the
first names of your instructional staff this summer! Please play with
this code before jumping into the roof material data if you are unsure
how to approach the question using \texttt{OneHotEncoder}.

\begin{verbatim}
>>> oh_enc = OneHotEncoder()
>>> oh_enc.fit([['Anirudhan'], ['Dominic'], ['Rahul'], ['Rahul'], ['Anirudhan'], ['Yike'], ['Vasanth']]);
>>> oh_enc.transform([['Anirudhan'], ['Rahul'], ['Dominic']]).toarray()
array([[1., 0., 0., 0., 0.],
       [0., 0., 1., 0., 0.],
       [0., 1., 0., 0., 0.]])
\end{verbatim}

    \begin{tcolorbox}[breakable, size=fbox, boxrule=1pt, pad at break*=1mm,colback=cellbackground, colframe=cellborder]
\prompt{In}{incolor}{43}{\boxspacing}
\begin{Verbatim}[commandchars=\\\{\}]
\PY{k+kn}{from} \PY{n+nn}{sklearn}\PY{n+nn}{.}\PY{n+nn}{preprocessing} \PY{k+kn}{import} \PY{n}{OneHotEncoder}

\PY{k}{def} \PY{n+nf}{ohe\PYZus{}roof\PYZus{}material}\PY{p}{(}\PY{n}{data}\PY{p}{)}\PY{p}{:}
\PY{+w}{    }\PY{l+s+sd}{\PYZdq{}\PYZdq{}\PYZdq{}}
\PY{l+s+sd}{    One\PYZhy{}hot\PYZhy{}encodes roof material.  New columns are of the form x0\PYZus{}MATERIAL.}
\PY{l+s+sd}{    \PYZdq{}\PYZdq{}\PYZdq{}}
    \PY{n}{oh\PYZus{}enc} \PY{o}{=} \PY{n}{OneHotEncoder}\PY{p}{(}\PY{p}{)}
    \PY{n}{a} \PY{o}{=} \PY{n}{oh\PYZus{}enc}\PY{o}{.}\PY{n}{fit\PYZus{}transform}\PY{p}{(}\PY{n}{data}\PY{p}{[}\PY{p}{[}\PY{l+s+s1}{\PYZsq{}}\PY{l+s+s1}{Roof Material}\PY{l+s+s1}{\PYZsq{}}\PY{p}{]}\PY{p}{]}\PY{p}{)}\PY{o}{.}\PY{n}{toarray}\PY{p}{(}\PY{p}{)}
    \PY{n}{df} \PY{o}{=} \PY{n}{pd}\PY{o}{.}\PY{n}{DataFrame}\PY{p}{(}\PY{n}{a}\PY{p}{,} \PY{n}{columns} \PY{o}{=} \PY{n}{oh\PYZus{}enc}\PY{o}{.}\PY{n}{get\PYZus{}feature\PYZus{}names}\PY{p}{(}\PY{p}{[}\PY{l+s+s1}{\PYZsq{}}\PY{l+s+s1}{Roof Material}\PY{l+s+s1}{\PYZsq{}}\PY{p}{]}\PY{p}{)}\PY{p}{)}
    \PY{k}{return} \PY{n}{df}\PY{o}{.}\PY{n}{merge}\PY{p}{(}\PY{n}{data}\PY{o}{.}\PY{n}{reset\PYZus{}index}\PY{p}{(}\PY{p}{)}\PY{p}{,} \PY{n}{left\PYZus{}index} \PY{o}{=} \PY{k+kc}{True}\PY{p}{,} \PY{n}{right\PYZus{}index} \PY{o}{=} \PY{k+kc}{True}\PY{p}{)}\PY{o}{.}\PY{n}{set\PYZus{}index}\PY{p}{(}\PY{l+s+s1}{\PYZsq{}}\PY{l+s+s1}{index}\PY{l+s+s1}{\PYZsq{}}\PY{p}{)}
\PY{n}{ohe\PYZus{}roof\PYZus{}material}\PY{p}{(}\PY{n}{training\PYZus{}data}\PY{p}{)}\PY{o}{.}\PY{n}{shape}
\PY{n}{training\PYZus{}data} \PY{o}{=} \PY{n}{ohe\PYZus{}roof\PYZus{}material}\PY{p}{(}\PY{n}{training\PYZus{}data}\PY{p}{)}
\PY{n}{training\PYZus{}data}\PY{o}{.}\PY{n}{filter}\PY{p}{(}\PY{n}{regex}\PY{o}{=}\PY{l+s+s1}{\PYZsq{}}\PY{l+s+s1}{\PYZca{}Roof Material\PYZus{}}\PY{l+s+s1}{\PYZsq{}}\PY{p}{)}\PY{o}{.}\PY{n}{head}\PY{p}{(}\PY{l+m+mi}{10}\PY{p}{)}
\end{Verbatim}
\end{tcolorbox}

            \begin{tcolorbox}[breakable, size=fbox, boxrule=.5pt, pad at break*=1mm, opacityfill=0]
\prompt{Out}{outcolor}{43}{\boxspacing}
\begin{Verbatim}[commandchars=\\\{\}]
       Roof Material\_Other  Roof Material\_Shake  \textbackslash{}
index
1                      0.0                  0.0
2                      0.0                  0.0
3                      0.0                  0.0
4                      0.0                  0.0
6                      0.0                  0.0
7                      0.0                  0.0
8                      0.0                  0.0
9                      0.0                  0.0
10                     0.0                  0.0
11                     0.0                  0.0

       Roof Material\_Shingle/Asphalt  Roof Material\_Slate  \textbackslash{}
index
1                                1.0                  0.0
2                                1.0                  0.0
3                                1.0                  0.0
4                                1.0                  0.0
6                                1.0                  0.0
7                                1.0                  0.0
8                                0.0                  0.0
9                                1.0                  0.0
10                               1.0                  0.0
11                               1.0                  0.0

       Roof Material\_Tar\&Gravel  Roof Material\_Tile
index
1                           0.0                 0.0
2                           0.0                 0.0
3                           0.0                 0.0
4                           0.0                 0.0
6                           0.0                 0.0
7                           0.0                 0.0
8                           1.0                 0.0
9                           0.0                 0.0
10                          0.0                 0.0
11                          0.0                 0.0
\end{Verbatim}
\end{tcolorbox}
        
    \begin{tcolorbox}[breakable, size=fbox, boxrule=1pt, pad at break*=1mm,colback=cellbackground, colframe=cellborder]
\prompt{In}{incolor}{44}{\boxspacing}
\begin{Verbatim}[commandchars=\\\{\}]
\PY{n}{grader}\PY{o}{.}\PY{n}{check}\PY{p}{(}\PY{l+s+s2}{\PYZdq{}}\PY{l+s+s2}{q7b}\PY{l+s+s2}{\PYZdq{}}\PY{p}{)}
\end{Verbatim}
\end{tcolorbox}

            \begin{tcolorbox}[breakable, size=fbox, boxrule=.5pt, pad at break*=1mm, opacityfill=0]
\prompt{Out}{outcolor}{44}{\boxspacing}
\begin{Verbatim}[commandchars=\\\{\}]
q7b results: All test cases passed!
\end{Verbatim}
\end{tcolorbox}
        
    \hypertarget{congratulations-you-have-finished-project-1a}{%
\subsection{Congratulations! You have finished Project
1A!}\label{congratulations-you-have-finished-project-1a}}

In Project 1B, you will focus on building a linear model to predict home
prices. You will be well-prepared to build such a model: you have
considered what is in this data set, what it can be used for, and
engineered some features that should be useful for prediction. Creating
a house-pricing model for Cook County has some challenging social
implications to think, though, however. This will be addressed in
Lecture 14 on July 14 (pretty cool coincidence?!) and Thursday's
discussion.

    \hypertarget{submission}{%
\subsection{Submission}\label{submission}}

Make sure you have run all cells in your notebook in order before
running the cell below, so that all images/graphs appear in the output.
The cell below will generate a zip file for you to submit.
\textbf{Please save before exporting!}

    \begin{tcolorbox}[breakable, size=fbox, boxrule=1pt, pad at break*=1mm,colback=cellbackground, colframe=cellborder]
\prompt{In}{incolor}{45}{\boxspacing}
\begin{Verbatim}[commandchars=\\\{\}]
\PY{c+c1}{\PYZsh{} Save your notebook first, then run this cell to export your submission.}
\PY{n}{grader}\PY{o}{.}\PY{n}{export}\PY{p}{(}\PY{n}{run\PYZus{}tests}\PY{o}{=}\PY{k+kc}{True}\PY{p}{)}
\end{Verbatim}
\end{tcolorbox}

    \begin{Verbatim}[commandchars=\\\{\}]
Running your submission against local test cases{\ldots}

Your submission received the following results when run against available test
cases:

    q2b results: All test cases passed!

    q3a results: All test cases passed!

    q3b results: All test cases passed!

    q4 results: All test cases passed!

    q5a results: All test cases passed!

    q5b results: All test cases passed!

    q6a results: All test cases passed!

    q6b results: All test cases passed!

    q6d results: All test cases passed!

    q6e results: All test cases passed!

    q7a results: All test cases passed!

    q7b results: All test cases passed!
    \end{Verbatim}

    
    \begin{Verbatim}[commandchars=\\\{\}]
<IPython.core.display.HTML object>
    \end{Verbatim}

    
    

    \begin{tcolorbox}[breakable, size=fbox, boxrule=1pt, pad at break*=1mm,colback=cellbackground, colframe=cellborder]
\prompt{In}{incolor}{ }{\boxspacing}
\begin{Verbatim}[commandchars=\\\{\}]

\end{Verbatim}
\end{tcolorbox}


    % Add a bibliography block to the postdoc
    
    
    
\end{document}
